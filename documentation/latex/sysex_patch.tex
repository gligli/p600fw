Patch parameters are stored as patches in each of the 100 slots on the Teensy++ 2.0 board (see section on memory usage). Apart from technical overhead this covers:
\begin{itemize}
  \item panel control settings
  \item menu parameters
  \item special data: per note tuning
  \item special data: note pattern (when the patch has unison activated with a chord pattern, this data contains the latched pattern) 
  \item special data: patch name 
\end{itemize}

The SysEx MIDI data is build directly from the storage data, e.g. it has the same structure (order of parameters). 

\subsection{Storage version}

Due to extensions of the parameter scope for additional functions, the patch storage structure depends on the firmware version. The storage versioning is as follows:

\begin{itemize}
  \item Firmware version 2.0: storage version 2
  \item Firmware version 2.1 RC3: storage version 7
  \item Firmware version \version: storage version 8  
\end{itemize}

For the implementation of an MIDI SysEx based patch editor it is crucial to take storage version into account. It is not recommended to try to convert a lower version SysEx file to a higher version, as several parameters have the value ranges changed. Instead one should always import a lower version SysEx patch file to the Prophet-600 with a newer firmware installed that uses the storage version you want and then export it again. The exported SysEx patch file will be properly converted. 

Note: it would be possible to document the transformations / remappings between versions 7 and 8. If you feel you need this for what you are trying to do, please get in contact via https://github.com/image-et-son/p600fw.

\subsection{Patch MIDI SysEx structure}

The SysEx structure for patch data is as follows:

\begin{enumerate}
  \item "F0": SysEx start
  \item Prophet-600 SysEx command signature: "00 61 16"
  \item Prophet-600 SysEx patch dump type: "01"
  \item Patch data section: series of MIDI data blocks, each consisting of 5 MIDI bytes (14 bit) 
  \item "F7": SysEx end
\end{enumerate}

The Prophet-600 ignores SysEx commands which do not carry the signature "00 61 16" with the exception of original Z80 produced SysEx files.

The data part of the patch SysEx has the same underlying structure as the internal storage in the sense that the parameters are stored in exactly the same order. As described in section \ref{midibyteconversion} the storage of double  byte values is added to the MIDI byte sequence in the order \textit{LSB}, \textit{MSB}. E.g the lower byte comes first. This also applies if a double byte is split across two MIDI data blocks (e.g. the \textit{LSB} is the fourth byte of one MIDI data block and the \textit{MSB} is the first byte of the next MIDI data block. 

The technical overhead inside the patch data section (item 4 in the list above) consists of another specific signature (\textit{Storage Magic}) and the stroage version. The signature is used to identify whether the data section of the SysEx represents a valid storage page of the Prophet-600. It should  always be present for any valid patch data independently from the storage version. If that signature is not present or differs the patch data will be ignored. The storage version should be used to identify the correct mapping of values.  

Note that the MIDI SysEx patch data contains all parameters up to and including the last non-zero value. It therefore has variable a size. However, it always consists of complete MIDI blocks (5 MIDI bytes which convert into 4 data bytes).

Each patch SysEx command as shown above represents exactly one flash storage page on the Teensy++ 2.0 board. A patch bank dump simply consists of a sequence of closed SysEx patch commands.

\subsection{Patch storage structure}\label{patchstore}

The following table specifies the patch data storage up to and including storage version 8. 

\footnotesize
\renewcommand{\arraystretch}{1.3}

\begin{longtable}[l]{p{5cm}|p{2cm}|p{1.5cm}|p{1.5cm}|p{2cm}|p{2.5cm}} 
\textbf{Patch Data} & \textbf{Type} & \textbf{Byte} & \textbf{Count} & \textbf{Total Bytes} & \textbf{From Version} \\ \hline
\endfirsthead
\textbf{Patch Data} & \textbf{Type} & \textbf{Byte} & \textbf{Count} & \textbf{Total Bytes} & \textbf{From Version} \\ \hline
\endhead 
Patch Number & Technical & 1 & 1 & 1 & 0 \\ \hline
Storage Key (aka MAGIC) & Technical & 4 & 1 & 4 & 0 \\ \hline
Storage Version & Technical & 1 & 1 & 1 & 0 \\ \hline
Fequency A & Continuous & 2 & 1 & 2 & 1 \\ \hline
Volume A & Continuous & 2 & 1 & 2 & 1 \\ \hline
PWA & Continuous & 2 & 1 & 2 & 1 \\ \hline
Fequency B & Continuous & 2 & 1 & 2 & 1 \\ \hline
Volume B & Continuous & 2 & 1 & 2 & 1 \\ \hline
PWB & Continuous & 2 & 1 & 2 & 1 \\ \hline
Frequency Fine B & Continuous & 2 & 1 & 2 & 1 \\ \hline
Cutoof & Continuous & 2 & 1 & 2 & 1 \\ \hline
Resonance & Continuous & 2 & 1 & 2 & 1 \\ \hline
Filter Envelope Amount & Continuous & 2 & 1 & 2 & 1 \\ \hline
Filter Release & Continuous & 2 & 1 & 2 & 1 \\ \hline
Filter Sustain & Continuous & 2 & 1 & 2 & 1 \\ \hline
Filter Decay & Continuous & 2 & 1 & 2 & 1 \\ \hline
Filter Attack & Continuous & 2 & 1 & 2 & 1 \\ \hline
Amp Release & Continuous & 2 & 1 & 2 & 1 \\ \hline
Amp Sustain & Continuous & 2 & 1 & 2 & 1 \\ \hline
Amp Decay & Continuous & 2 & 1 & 2 & 1 \\ \hline
Amp Attack & Continuous & 2 & 1 & 2 & 1 \\ \hline
Poly Mod Envelope Amount & Continuous & 2 & 1 & 2 & 1 \\ \hline
Poly Mod OSC B & Continuous & 2 & 1 & 2 & 1 \\ \hline
LFO Frequency & Continuous & 2 & 1 & 2 & 1 \\ \hline
LFO Amount & Continuous & 2 & 1 & 2 & 1 \\ \hline
Glide & Continuous & 2 & 1 & 2 & 1 \\ \hline
Amp Velocity & Continuous & 2 & 1 & 2 & 1 \\ \hline
Filter Velocity & Continuous & 2 & 1 & 2 & 1 \\ \hline
Saw A & Stepped & 1 & 1 & 1 & 1 \\ \hline
Tri A & Stepped & 1 & 1 & 1 & 1 \\ \hline
SQR A & Stepped & 1 & 1 & 1 & 1 \\ \hline
Saw B & Stepped & 1 & 1 & 1 & 1 \\ \hline
Tri B & Stepped & 1 & 1 & 1 & 1 \\ \hline
SQR B & Stepped & 1 & 1 & 1 & 1 \\ \hline
Sync & Stepped & 1 & 1 & 1 & 1 \\ \hline
Poly Mod Frequency A & Stepped & 1 & 1 & 1 & 1 \\ \hline
Poly Mod Filter & Stepped & 1 & 1 & 1 & 1 \\ \hline
LFO Shape & Stepped & 1 & 1 & 1 & 1 \\ \hline
(unused) & n/a  & 1 & 1 & 1 & n/a\footnote{This slot was used for \textit{LFO Frequency Range} from versions 1 to 7} \\ \hline
LPF Targets & Stepped & 1 & 1 & 1 & 1 \\ \hline
Tracking Shift & Stepped & 1 & 1 & 1 & 1 \\ \hline
Filter Envelope Shape & Stepped & 1 & 1 & 1 & 1 \\ \hline
Filter Envelope Speed & Stepped & 1 & 1 & 1 & 1 \\ \hline
Amp Envelope Shape & Stepped & 1 & 1 & 1 & 1 \\ \hline
Amp Envelope Speed & Stepped & 1 & 1 & 1 & 1 \\ \hline
Unison & Stepped & 1 & 1 & 1 & 1 \\ \hline
Assigner Priority & Stepped & 1 & 1 & 1 & 1 \\ \hline
Bender Semitones & Stepped & 1 & 1 & 1 & 1 \\ \hline
Bender Target & Stepped & 1 & 1 & 1 & 1 \\ \hline
Mod Wheel Shift & Stepped & 1 & 1 & 1 & 1 \\ \hline
Chromatic Pitch & Stepped & 1 & 1 & 1 & 1 \\ \hline
Modulation Delay & Continuous & 2 & 1 & 2 & 2 \\ \hline
Vibrato Frequency & Continuous & 2 & 1 & 2 & 2 \\ \hline
Vibrato Amount & Continuous & 2 & 1 & 2 & 2 \\ \hline
Unison Detune & Continuous & 2 & 1 & 2 & 2 \\ \hline
(unused) & n/a & 2 & 1 & 2 & n/a\footnote{This slot was used for arpeggiator/sequencer clock speed fro versions 1 to 7. From version 8 on clock speed is consistently implemented as a setting only.} \\ \hline
Modulation Wheel Target & Stepped & 1 & 1 & 1 & 2 \\ \hline
Vibrato Target & Stepped & 1 & 1 & 1 & 2 \\ \hline
Voice Pattern (6 voices) & Special & 1 & 6 & 6 & 2 \\ \hline
Tuning per Note (12 notes) & Special & 2 & 12 & 24 & 7 \\ \hline
Sync Bug & Stepped & 1 & 1 & 1 & 8 \\ \hline
Vintage (Spread) & Continuous & 2 & 1 & 2 & 8 \\ \hline
External CV & Continuous & 2 & 1 & 2 & 8 \\ \hline
Envelope Routing & Stepped & 1 & 1 & 1 & 8 \\ \hline
Voice Assign & Stepped & 1 & 1 & 1 & 8 \\ \hline
LFO Sync & Stepped & 1 & 1 & 1 & 8 \\ \hline
Patch Name & Technical & 1 & 16 & 16 & 8 \\
\end{longtable}

\textbf{Example}

Consider building the SysEx patch files for patch number 5 from storage in the latest version 8. The first 8 data bytes prior to conversion are: "05 00 61 16 A5 08 00 FF", where 05 is the patch number, "00 61 16 A5" is the \textit{Storage Magic}, "08" is the storage version and "00 FF" is an example value for the first real patch parameter which is frequency of oscillator A. These 8 data bytes are first rearranged into two blocks of 4 bytes like: "05 A5 16 61" "00 08 FF 00". Note that the order of bytes for each data block is reversed with \textit{LSB} first, etc. Then the conversion to 5 MIDI bytes leads to two blocks of 5 MIDI bytes like this: "05 25 16 61 02" "00 08 7F 00 04".

\subsection{Patch Name}

Up to version \version the Prophet-600 does no display the patch name and offers no option the change the patch name. However, the patch name data is stored in the patch and is preserved when the patch is send to Prophet-600 via MIDI SysEx, then stored on the unit and then exported again in SysEx format. An external SysEx patch editor could therefore make use of the patch name.

The patch consists of 16 bytes and the enconding is UTF-8 with printable characters as follows:

\begin{itemize}
  \item 00: NULL character terminates the patch name
  \item 01 - 0F: non-printable characters are interpreted as \textit{space}
  \item 20 - 7E: standard UTF-8 printable charatcers from \textit{space} through number, letters and special characters up to \textasciitilde
  \item 7F - FF: DEL and following values are interpreted as a \textit{space}
\end{itemize}
 

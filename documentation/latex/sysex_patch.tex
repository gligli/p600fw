Patch parameters are stored as patches in each of the 100 slots on the Teensy 2.0++ board (see section on memory usage). Apart from technical overhead this covers:
\begin{itemize}
  \item panel control settings
  \item menu parameters
  \item special date: per note tuning
  \item special data: note pattern (when the patch has unison activated with a chord pattern this contains the pattern) 
\end{itemize}
The SysEx MIDI data is build directly from the stoage data, e.g. it has the same structure (order of parameters). 

\subsection{Storage version}

\begin{itemize}
  \item Firmware version 2.0: storage version 2
  \item Firmware version 2.1 RC3: storage version 7
  \item Firmware version \version: storage version 8  
\end{itemize}


\subsection{Patch storage structure}

The following table specifies the patch data storage up to and including storage version 8. 

\footnotesize
\renewcommand{\arraystretch}{1.3}

\begin{longtable}[l]{|p{5cm}|p{2cm}|p{1.5cm}|p{1.5cm}|p{1.5cm}|p{1.5cm}|} 
\hline
\textbf{Patch Data} & \textbf{Type} & \textbf{Byte} & \textbf{Count} & \textbf{Total Bytes} & \textbf{From Version} \\ \hline
Storage Key (aka MAGIC) & Technical & 4 & 1 & 4 & 0 \\ \hline
Storage Version & Technical & 1 & 1 & 1 & 0 \\ \hline
Fequency A & Continuous & 2 & 1 & 2 & 1 \\ \hline
Volume A & Continuous & 2 & 1 & 2 & 1 \\ \hline
PWA & Continuous & 2 & 1 & 2 & 1 \\ \hline
Fequency B & Continuous & 2 & 1 & 2 & 1 \\ \hline
Volume B & Continuous & 2 & 1 & 2 & 1 \\ \hline
PWB & Continuous & 2 & 1 & 2 & 1 \\ \hline
Frequency Fine B & Continuous & 2 & 1 & 2 & 1 \\ \hline
Cutoof & Continuous & 2 & 1 & 2 & 1 \\ \hline
Resonance & Continuous & 2 & 1 & 2 & 1 \\ \hline
Filter Envelope Amount & Continuous & 2 & 1 & 2 & 1 \\ \hline
Filter Release & Continuous & 2 & 1 & 2 & 1 \\ \hline
Filter Sustain & Continuous & 2 & 1 & 2 & 1 \\ \hline
Filter Decay & Continuous & 2 & 1 & 2 & 1 \\ \hline
Filter Attack & Continuous & 2 & 1 & 2 & 1 \\ \hline
Amp Release & Continuous & 2 & 1 & 2 & 1 \\ \hline
Amp Sustain & Continuous & 2 & 1 & 2 & 1 \\ \hline
Amp Decay & Continuous & 2 & 1 & 2 & 1 \\ \hline
Amp Attack & Continuous & 2 & 1 & 2 & 1 \\ \hline
Poly Mod Envelope Amount & Continuous & 2 & 1 & 2 & 1 \\ \hline
Poly Mod OSC B & Continuous & 2 & 1 & 2 & 1 \\ \hline
LFO Frequency & Continuous & 2 & 1 & 2 & 1 \\ \hline
LFO Amount & Continuous & 2 & 1 & 2 & 1 \\ \hline
Glide & Continuous & 2 & 1 & 2 & 1 \\ \hline
Amp Velocity & Continuous & 2 & 1 & 2 & 1 \\ \hline
Filter Velocity & Continuous & 2 & 1 & 2 & 1 \\ \hline
Saw A & Stepped & 1 & 1 & 1 & 1 \\ \hline
Tri A & Stepped & 1 & 1 & 1 & 1 \\ \hline
SQR A & Stepped & 1 & 1 & 1 & 1 \\ \hline
Saw B & Stepped & 1 & 1 & 1 & 1 \\ \hline
Tri B & Stepped & 1 & 1 & 1 & 1 \\ \hline
SQR B & Stepped & 1 & 1 & 1 & 1 \\ \hline
Sync & Stepped & 1 & 1 & 1 & 1 \\ \hline
Poly Mod Frequency A & Stepped & 1 & 1 & 1 & 1 \\ \hline
Poly Mod Filter & Stepped & 1 & 1 & 1 & 1 \\ \hline
LFO Shape & Stepped & 1 & 1 & 1 & 1 \\ \hline
LFO Sync & Stepped & 1 & 1 & 1 & 8\footnote{This slot was used for \textit{LFO Frequency Range} from versions 1 to 7} \\ \hline
LPF Targets & Stepped & 1 & 1 & 1 & 1 \\ \hline
Tracking Shift & Stepped & 1 & 1 & 1 & 1 \\ \hline
Filter Envelope Shape & Stepped & 1 & 1 & 1 & 1 \\ \hline
Filter Envelope Speed & Stepped & 1 & 1 & 1 & 1 \\ \hline
Amp Envelope Shape & Stepped & 1 & 1 & 1 & 1 \\ \hline
Amp Envelope Speed & Stepped & 1 & 1 & 1 & 1 \\ \hline
Unison & Stepped & 1 & 1 & 1 & 1 \\ \hline
Assigner Priority & Stepped & 1 & 1 & 1 & 1 \\ \hline
Bender Semitones & Stepped & 1 & 1 & 1 & 1 \\ \hline
Bender Target & Stepped & 1 & 1 & 1 & 1 \\ \hline
Mod Wheel Shift & Stepped & 1 & 1 & 1 & 1 \\ \hline
Chromatic Pitch & Stepped & 1 & 1 & 1 & 1 \\ \hline
Modulation Delay & Continuous & 2 & 1 & 2 & 2 \\ \hline
Vibrato Frequency & Continuous & 2 & 1 & 2 & 2 \\ \hline
Vibrato Amount & Continuous & 2 & 1 & 2 & 2 \\ \hline
Unison Detune & Continuous & 2 & 1 & 2 & 2 \\ \hline
(unused) & n/a & 2 & 1 & 2 & n/a\footnote{This slot was used for arpeggiator/sequencer clock speed fro versions 1 to 7. From version 8 on clock speed is consistently implemented as a setting only.} \\ \hline
Modulation Wheel Target & Stepped & 1 & 1 & 1 & 2 \\ \hline
Vibrato Target & Stepped & 1 & 1 & 1 & 2 \\ \hline
Voice Pattern (6 voices) & Special & 1 & 6 & 6 & 2 \\ \hline
Tuning per Note (12 notes) & Special & 2 & 12 & 24 & 7 \\ \hline
Sync Bug & Stepped & 1 & 1 & 1 & 8 \\ \hline
Vintage (Spread) & Continuous & 2 & 1 & 2 & 8 \\ \hline
External CV & Continuous & 2 & 1 & 2 & 8 \\ \hline
Envelope Routing & Stepped & 1 & 1 & 1 & 8 \\ \hline
Voice Assign & Stepped & 1 & 1 & 1 & 8 \\ \hline
Patch Name & Technical & 1 & 16 & 16 & 8 \\ \hline
\end{longtable}

\subsection{Patch MIDI SysEx structure}

The SysEx structure for patch is as follows:

\begin{itemize}
  \item "F0": SysEx start
  \item Prophet 600 signature: "00 61 16"
  \item Prophet 600 SysEx patch dump type: "01"
  \item Patch data section: series of MIDI data blocks, eachconsisting of 5 MIDI bytes (14 bit) 
  \item 3 bytes of checksum
  \item "F7": SysEx end
\end{itemize}

The data part of the patch SysEx has the same underlying structure as the internal storage in the sense that the parameters are stored in exactly the same order. As described in section \ref{midibyteconversion} the storage of double  byte values ist added to the MIDI byte sequence in the order \textit{LSB}, \textit{MSB}. E.g the lower byte comes first. This also applies if a double byte is split across two MIDI data blocks (e.g. the \textit{LSB} is the fourth byte of one MIDI data block and the \textit{MSB} is the first byte of the next MIDI data block. 

The technical overhead inside the data section consists of a specific key (\textit{storage magic}) and the stroage version. The key can be used to identify whether the data section of the SysEx represents a storage page of the Prophet 600. It should  always be present for any valid patch data independently from the storage version. The storage version should be used to identify the correct mapping of values.  

Note that the MIDI SysEx patch data contains all parameters up to and including the last non-zero value. It there has variable a size.

\textbf{Checksum}

(this needs to be done)

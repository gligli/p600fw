\textbf{Power Switch}

When turning on the Prophet 600 starts in normal mode when no buttons held down. The Prophet 600 can be powered up in two different modes related to maintenance tasks: firmware upgrade mode (see section \ref{maintenance}) and scaling adjustment mode (see section \ref{fwupgrade}).  

When started in normal mode the specification of the installed version is scrolled through the display. Then, the display will show and remain on the patch number selected when the instrument was last powered off.

Note: earlier versions ofthe firmware upgrade went into tuning mode directly when powered up. From version 2.1RC3 on this is no longer the case. In practice, the instrument needs to warm up and stabilize before tuning makes sense. For using tuning see section \ref{tuning}.

\textbf{Line voltage selector}

The selector next to the main plug provides the option to switch between 110V and 220V mains voltage, whichever is appropriate for the line voltage where the instrument is operated. A flat head screw driver can be used to rotate the line voltage selector to the appropriate setting. Instrument should be powered down and unplugged when line voltage selector is adjusted.

\textbf{Fuse holder}

Fit the fuse according to the selected line voltage, e.g. 110V 1/2A, slo-blo, 220V 1/4A slo-blo.

\textbf{Audio out / headphone jack}

To drive a preamp or amp, a standard monophonic cable can be used. Or if there are two audio destinations (perhaps one to remain “dry” while the other is processed), a stereo cable may be more convenient.

The Prophet 600 has a monophonic output signal, but the jack is wired so that both sides of standard stereo headphones can be driven. The headphones have a minimum impedance of 1200 Ohms per element (600 Ohms in parallel).

\textbf{Filter CV In}

This jack accepts a 0-10 Vdc control voltage (CV) which is applied to the filter cutoff frequency (all voices). This enables remote and spontaneous increase (but not decrease) of the filter cutoff, e.g. making the sound brighter. The CV can be usually provided by an accessory voltage pedal or an external LFO. The strength of the external CV influence is controlled by an additional patch parameter (see section \ref{extcv}). The default value (cf. default patch) of this parameter is zero. 

\textbf{Foot switch}

The foot switch jack allows for the use of a sustain pedal (only supported from GliGli 2.1RC3 onwards). The pedal should be contact open at rest (for example a Fatar PS100 will function correctly).

\textbf{Cassette In/Out (disabled with GliGli mod)}

Cassette In can be used for sending a pulsed clock to the Prophet 600 to sync to the sequencer and/or arpeggiator. However, in the upgraded versionthe functionality to receive or send patch data via Cassette In/Out has been omitted. 

\textbf{MIDI In/Out}

The Prophet 600 is equipped with standard MIDI In and MIDI Out connections. The MIDI receive and send channels can be in the miscellaneous settings, see section \ref{midiintegration} on MIDI integration.  

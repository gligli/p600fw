Two types of MIDI events are implemented in the ,upgraded Prophet-600:

\begin{itemize}
  \setlength\itemsep{0cm}
  \item MIDI CC Continuous parameters: value resolution of 0-16383 using 2 CCs (fine and coarse), or value resolution of 0-127 using only the coarse one
  \item MIDI CC Stepped parameters: 0-127, variable number of steps. They work by dividing the 0-127 range in as many zones as there are choices for the parameter. E.g.: "Unison" is an on-off parameter and therefore has s choices. In this case it is \textit{off} for 0-63 and it is \textbf{on} for 64-127.
  \item Performance events (note events, pitch bend)
  \item Real time events
  \item SysEx (for different purposes)
\end{itemize}

The Prophet-600 receives Continuous Controllers (CC) which correspond to patch parameters in \presetmode only. 

\section{MIDI CC Patch Parameters} 

\footnotesize
\renewcommand{\arraystretch}{1.3}

\begin{longtable}[l]{ p{5cm}|p{2cm}|p{1.5cm}|p{1.5cm}|p{5cm}|p{2cm}|p{1cm}} 
\textbf{Continuous Parameter} & \textbf{Type} & \textbf{CC Coarse} & \textbf{CC Fine} & \textbf{Stepped Parameter} & \textbf{Type} & \textbf{CC} \\ \hline
\endfirsthead
\textbf{Continuous Parameter} & \textbf{Type} & \textbf{CC Coarse} & \textbf{CC Fine} & \textbf{Stepped Parameter} & \textbf{Type} & \textbf{CC} \\ \hline
\endhead 
Osc A Frequency & Continuous & 16 & 80 & Osc A Saw & Stepped & 48 \\ \hline
Osc A Volume & Continuous & 17 & 81 & Osc A Triangle & Stepped & 49 \\ \hline
Osc A Pulse Width & Continuous & 18 & 82 & Osc A Square & Stepped & 50 \\ \hline
Osc B Frequency & Continuous & 19 & 83 & Osc B Saw & Stepped & 51 \\ \hline
Osc B Volume & Continuous & 20 & 84 & Osc B Triangle & Stepped & 52 \\ \hline
Osc B Pulse Width & Continuous & 21 & 85 & Osc B Sqr & Stepped & 53 \\ \hline
Osc B Fine & Continuous & 22 & 86 & Sync & Stepped & 54 \\ \hline
Cutoff & Continuous & 23 & 87 & Poly Mod Oscillator A Destination & Stepped & 55 \\ \hline
Resonance & Continuous & 24 & 88 & Poly Mod Filter Destination & Stepped & 56 \\ \hline
Filter Envelope Amount & Continuous & 25 & 89 & LFO Shape & Stepped & 57 \\ \hline
Filter Release  &  Continuous  & 26 & 90 &  LFO Targets  &  (see \ref{lfotarget})  &  59 \\ \hline
Filter Sustain  &  Continuous  & 27 & 91 &  Keyboard Filter Tracking  &  Stepped  &  60 \\ \hline
Filter Decay  &  Continuous  & 28 & 92 &  Filter Envelope Shape  &  Stepped  &  61 \\ \hline
Filter Attack  &  Continuous  & 29 & 93 &  Filter Envelope Fast/Slow  &  Stepped  &  62 \\ \hline
2nd Release  &  Continuous  & 30 & 94 &  2nd Envelope Shape  &  Stepped  &  63 \\ \hline
2nd Sustain  &  Continuous  & 31 & 95 &  Unison  &  Stepped  &  65 \\ \hline
2nd Decay  &  Continuous  & 32 & 96 &  Assigner Priority Mode  &  Stepped  &  66 \\ \hline
2nd Attack  &  Continuous  & 33 & 97 &  Bender Range (semitones)  &  (see \ref{bendsemi})  &  67 \\ \hline
Poly Mod Filter Amount  &  Continuous  & 34 & 98 &  Bender Target  &  Stepped  &  68 \\ \hline
Poly Mod Osc B Amount  &  Continuous  & 35 & 99 &  Mod Wheel Range  &  Stepped  &  69 \\ \hline
LFO Frequency  &  Continuous  & 36 & 100 &  Osc pitch mode  &  Stepped  &  70 \\ \hline
LFO Amount  &  Continuous  & 37 & 101 &  Mod Wheel Target  &  Stepped  &  71 \\ \hline
Glide  &  Continuous  & 38 & 102 &  Vibrato Target  &  Stepped  &  72 \\ \hline
Amp Velocity  &  Continuous  & 39 & 103 &   2nd Envelope Fast/Slow  &  Stepped  &  73 \\ \hline
Filter Velocity  &  Continuous  & 40 & 104 &  Sync Bug  &  Stepped  &  74 \\ \hline
Modulation delay  &  Continuous  & 41 & 105 &  Voice Assign  &  Stepped  &  75 \\ \hline
Vibrato frequency  &  Continuous  & 42 & 106 &  Envelope Routing  &  Stepped  &  76 \\ \hline
Vibrato amount  &  Continuous  & 43 & 107 &  LFO Sync  &  Stepped  &  77 \\ \hline
Unison detune  &  Continuous  & 44 & 108 &    &    &   \\ \hline
External CV amount  &  Continuous  & 46 & 110 &  &  &   \\ \hline 
Vintage / Spread  &  Continuous  & 47 & 111 &  &  &   \\ \hline 
 
\end{longtable}

\normalsize

\section{MIDI CC Settings}

Unlike MIDI CC event for patch parameters, which are only applied in \presetmode, the following MIDI CC controls are always applied. 

\renewcommand{\arraystretch}{1.3}

\begin{longtable}[l]{ p{6cm}|p{2.5cm}|p{1.5cm}|p{2cm}|p{7cm}} 
\textbf{Parameter} & \textbf{Type} & \textbf{CC} & \textbf{CC Fine} & \textbf{Comment on CC value} \\ \hline
Toggle Live/Preset Mode & Stepped & 0 & & 0 = live mode, 1 = preset mode \\ \hline
Mod Wheel & Continuous & 1 & & \\ \hline
Master Volume & Continuous & 7 & & \\ \hline
Sequencer / arpeggiator clock & Continuous & 45 & 109 & \\ \hline
Hold Pedal & Stepped & 64 & & 0 = release, >0 = hold/latch\\ \hline
All Sound Off & Stepped & 120 & & 127 = sound off\\ \hline
Local On/Off & Stepped & 122 & & <=63 = local on, >63 = local off \\ \hline
All Notes Off & Stepped & 123 & & 0 = all notes off\\ \hline
\end{longtable}

\section{Special parameters and technical information}

\subsection*{LFO Targets (MIDI CC 59)}\label{lfotarget}

MIDI CC 59 LFO Target is a stepped parameter. It combines the effects of the LFO panel controls (separate switches for targeting the VCO, the filter cutoff and the pulse width) with the additional patch parameter which provides further options (targeting VCA, targeting oscillators A and B separately). These options are encoded bitwise in the parameter as shown in the table below. Each bit represents \textit{off} or \textit{on} for the respective switch/option. 

\renewcommand{\arraystretch}{1.3}

\begin{longtable}[l]{p{1.2cm}|p{2cm}|p{4cm}} 
\textbf{Bit} & \textbf{Value} & \textbf{Function} \\ \hline
\endfirsthead
\textbf{Bit} & \textbf{Value} & \textbf{Function} \\ \hline
\endhead 
1 & 1 & VCO \\ \hline 
2 & 2 & Filter \\ \hline 
3 & 4 & VCA \\ \hline 
4 & 8 & Pulse Width \\ \hline 
5 & 16 & Only A \\ \hline 
6 & 32& Only B \\ \hline 
\end{longtable}


Example: if the filter and the VCO targets are active and the modulation is restricted to oscillator B then the parameter takes the following value:

LFO Targets (MIDI CC 59) = 1 (VCO) + 2 (filter) + 32 (only B) = 35

Note that the same parameter encoding is also used in the patch MIDI SysEx implementation. Not all combinations of bits are possible (at least when edited on the unit) and meaningful. Some can only be achieved using MIDI CC or SysEx.
\begin{itemize}
  \item The options "Only A" and "Only B" only affect the targets VCO and pulse width (VCA and filter are shared by the two oscillators)
  \item When editing a patch from the Prophet-600 it is impossible to set bits 5 and 6 simultaneously. Technically the effect of setting both bits (by sending such a patch configuration to the Prophet-600 via SysEx or MIDI CC) would be that VCO and pulse width are modulated for neither oscillator even if those targets are activated. 
  \item When editing a patch from the Prophet-600 it is impossible to activate the VCA target and at the same time restrict the VCO or pulse width modulation to one of the oscillators. This is purely due to user interface restrictions. It is perfectly possible to choose this option using MIDI CC or SysEx. 
\end{itemize}


\subsection*{Bender Range (semitones) (MIDI CC 67)}\label{bendsemi}

The MIDI CC 67 parameter which controls the bender range behaves like a stepped parameter. However, unlike other stepped parameters whose value corresponds to the index of the selected option, the value of the bend range parameter contains the actual number of semitones. The allowed values and the corresponding option names as shown on the Prophet-600 display are as follows. 

\renewcommand{\arraystretch}{1.3}

\begin{longtable}[l]{p{1.2cm}|p{4cm}} 
\textbf{Value} & \textbf{Text} \\ \hline
\endfirsthead
\textbf{Value} & \textbf{Text} \\ \hline
\endhead 
2 & 2nd \\ \hline 
3 & 3rd \\ \hline 
5 & 5th \\ \hline 
8 & Octave \\ \hline 
\end{longtable}

No other values are allowed.

\subsection*{Changes in MIDI CC implementation from version 2.1 RC3 to \version}

The following MIDI CC have been changed.

\begin{itemize}
  \item For the filter envelope release and decay and the 2nd envelope release and decay the time value range has been rescaled for the \textit{linear} envelope shape. This was an unavoidable side effect of changing the strictly linear shapes to linear shapes with soft tails. The exponential envelopes are unaffected.
  \item The \lfofreq has been totally rescaled. The stepped frequency range parameter has been omitted and the value range of the \lfofreq dial now covers the entire frequency range. However, the dial to frequency relation has also been changed from linear to exponential, making it much softer at low settings. Consequently, the MIDI CC support for the LFO frequency range (up to firmware version 2.1 RC3) has been omitted. In order to avoid potential conflicts with MIDI tools developed for firmware versions 2.0 and 2.1 RC3, the legacy MIDI CC 58 has not been re-designated. MIDI CC 58 is ignored.
  \item The value to frequency relation for the \vibspeed has been changed from linear to exponential.   
  \item The \modwheelrange has been reconfigured with new value which also produce different wheel actions.
\end{itemize}


\normalsize

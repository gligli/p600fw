You can play the upgraded Prophet-600 in the following voice modes
\begin{itemize}
  \setlength\itemsep{0cm}
  \item \textit{polyphonic}: This mode is set by switching \unison to \textit{off}. In this mode any new note is assigned to one of the 6 voices. This is the default patch setting.
  \item \textit{unison}: This mode is activated by switching \unison \textit{on} without holding any keyboard key pressed. In this mode all 6 voices play the same note. Apply the additional patch parameter \detune to create thickened sounds in unison mode, see section \ref{detune}.
  \item \textit{chord}: This mode is activated by switching \unison \textit{on} while holding down one or more keys (the "chord"). In this mode the synthesizer transposes that chord over the whole keyboard. Note that independently of how many notes the chord contains (and therefore how many voices would be left in principle), only one chord can be played at a time, similar to unison mode.
\end{itemize} 

\unison is a patch parameter -- i.e. it is stored with the patch.

Note that in unison/chord modes the foot switch does not hold/sustain notes as in polyphonic mode but instead latches patterns of notes / chords on-the-fly without the need to leave and reactivate the unison/chord mode. To use on-the-fly latching, hold down the chord you want to latch in unison/chord mode and press the foot switch. In order to return to normal unison mode, press the foot switch with no key held down. Using the foot switch with \unison on re-triggers the currently held notes. 

The assignment of voices to newly played notes can be customized using the additional patch parameter \prio. New notes will be assigned to voices according to the following priority rules:

\begin{itemize}
  \setlength\itemsep{0cm}
  \item \textit{last}: New notes always play and if 6 voices are already playing then the oldest played note is "stolen"
  \item \textit{low}: A newly played note only plays if it is lower than the highest note already playing. The highest note is "stolen". With \unison on, legato is active. This is the preferred setting when bass accompaniment is important and the sudden stop of bass notes would create unwanted disruptions. 
  \item \textit{high}: A newly played note only plays if it is higher than the lowest note already playing. The lowest note is "stolen". With \unison on, legato is active. This is the preferred setting for a synthesizer lead situation where the sudden stop of a high note would create unwanted disruptions. 
\end{itemize}

The assigner priority of voice mode is a patch parameter -- i.e. it is stored with the patch. 

Note on the return of "stolen" notes: when more than 6 notes are held in priority \textit{last}, only the newest 6 notes play. If one of those keys is released then this note is cut and a stolen note is re-assigned and will sound, i.e. it will re-appear. The note is not re-triggered in this case but rather takes over from the envelope positions of the note that was just released and cut. The note to re-appear is the oldest of the stolen notes. However, with the re-assignment this note becomes the newest assigned note. When another key is release the next note to re-appear will therefore be the one which originally the second oldest, and so forth. Glide is applied to re-appearing notes.

\textbf{Voice assignment logic}

As described above the assigner priority determines which notes are played or terminated once there are more notes to play than the available 6 voices (or less if one or more are deactivated). A separate part of the assigner logic is concerned with the selection of the target voice when a new note is to be assigned. There are two choices for this logic which you can toggle using the menu parameter \assign:

\begin{itemize}
  \item \textit{first}: the first free voice (in the sense of the fixed internal numbering of voices from 1 to 6) is selected. I.e. when only one note is played only the first voice is used.
  \item \textit{cycle}: also known as "round robin", this logic always selects a new voice in order to distribute the load evenly among the voices. I.e. when only one note is played each voice will be used in turn.
\end{itemize} 

In practise the first logic this means that when a note is released and played again the same voice is used. Since voices can have different characteristics in analogue synthesizer such as the Prophet-600, this type of voice allocation logic favours more stable and consistent note sequences. Using the second logic, on the other hand, you can make deliberate use of the variations in voice characteristics to create more lively and animated sequences. This becomes very pronounced when vintage spread is activated (see section \ref{spreadsett}). Note also that glide is based on voice, i.e. once a note is assigned to a voice the glide start note will be the last note of that voice (rather than, for example, the last played note). Therefore the parameter \assign also has a strong impact on the behaviour of glide in polyphonic mode. For glide see section \ref{glide}.

\textbf{Ideas for solo voices}

You can use the assigner priority effectively in combination with the voice modes to customize solo voices. For example combine the priority \textit{high} (or \textit{low}) with ...
\begin{itemize}
  \item ...\textit{unison} mode for a massive legato solo voice
  \item ...\textit{chord} mode with a one key "chord" (hold down only C3) to achieve a more delicate legato mono solo voice
  \item ...\textit{chord} mode with octaves (hold down C3 plus C4 plus more if desired) for a more assertive mono solo voice   
\end{itemize}


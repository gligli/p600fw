The upgraded Prophet-600 features a dedicated vibrato. Its parameters are available in the additional patch parameter menu as follows.

\begin{itemize}
  \item The \vibspeed sets the vibrato frequency in a range of approximately 1/20 to 60 Hz
  \item The \vibamt sets the vibrato strength
  \item The \vibtgt sets the vibrato target. The options are \textit{VCO} (modulating the pitch of oscillators A  and B), \textit{VCA} (modulating the amplitude) as well as \textit{VCO A} and \textit{VCO B}(modulating only the pitch of one oscillator A  or  B, respectively)
\end{itemize}

The vibrato shape is a triangle wave. The vibrato effect can be controlled by either the modulation wheel or a modulation delay function. The additional patch parameter \modwheeltarget can be set to \textit{vibrato} or \textit{LFO}. When set to \textit{vibrato}, pushing the \modwheel up adds vibrato strength. The strength of the \modwheel is controlled by the additional patch parameter \modwheelrange. If you set the \modwheel target to \textit{LFO}, on the other hand, the modulation delay is automatically applied to the vibrato, i.e. the vibrato onset is delayed by the amount set by the additional patch parameter \moddelay. For setting the parameters of the modulation wheel see section \ref{modwheel}. For complementary functions for LFO see section \ref{lfo}.

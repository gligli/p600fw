For supporting the integration of the Prophet 600 in a MIDI setting, e.g. using SysEx, integration in DAW setup or simply attaching an external controller, the MIDI send and receive channels can be set using dedicated miscellaneous settings as follows.

\begin{enumerate}
  \setlength\itemsep{0cm}
  \item Setting number 1: pushing key "1" repeatedly while holding down From Tape cycles through the MIDI receive channels, from \textit{OMNI} through channels \textit{Ch1} to \textit{Ch16}
  \item Setting number 2: pushing key "2" repeatedly while holding down From Tape cycles through the MIDI send channels, from \textit{Ch1} to \textit{Ch16}
\end{enumerate}


\textbf{Local off mode}

The Prophet 600 supports a \textit{local off} mode which is toggled using the setting "7". In \textit{local off} the following internal events are not executed in the Prophet 600 but are still sent as MIDI events. The typical setup would be to connect to a MIDI recorder and feed the recorded MIDI messages back to the instrument for replay. The \textit{local off} mode ensures that there is no doubling of events.

The following MIDI events can be recorded and are executed by the Prophet 600 when played back to it.

\begin{enumerate}
  \setlength\itemsep{0cm}
  \item Key on, key off
  \item Pitch Bender
  \item Modulation Wheel
  \item Hold Pedal (in polyphonic mode, not in unison/chord mode or running arpeggiator or sequencer)
\end{enumerate}

Note that program changes are sent via MIDI but are executed in the Prophet 600 also in \textit{local off} mode. The Prophet 600 accept and executes program change MIDI messages only in \textit{preset mode}.  

\textbf{Overview of MIDI integration}


The Prophet-600 is a synthesizer using microchips that implement voltage controlled oscillators. This means that the oscillator frequency is set by applying a voltage to the control pin of a microchip. The relationship between absolute voltage and resulting frequency is not precise. It depends on many factors, such as smallest differences in the manufacturing process of the chip itself and the surrounding circuit components and operating temperature (the temperature of the components also changes during operation). Therefore such synthesizers need to be tuned regularly.  The Prophet-600 has a tuning procedure which determines the correct voltages which need to be applied to oscillators A and B and to the filter frequency control in order to achieve a perfect pitch. There is one measurement point per octave over several octaves (over 8 octaves in principle, even though the lowest and highest are extrapolated). When a note is played, the voltage to achieve the corresponding pitch is interpolated. This is how the Prophet-600 stays in tune. 

The tuning data is stored in the settings so it is readily available after the instrument is switched on. However, while the tuning primarily takes into account the individual characteristics of the chips one should be aware that voltage-to-pitch characteristics depend on different factors, notably on temperature, and can therefore "drift". Therefore the tuning procedure must be started manually by pressing the \tune button (while not in \shiftmode or \shiftlock as pressing \tune in this case activates the per note tuning mode, see below) when there is the need to do so. Typically, tuning is best after the instrument (or rather: the chips and components) has been given time to warm up to a stable operating temperature, which could be 10-30 minutes into operation. The warm up time can depend on changes in air temperature, humidity and transportation of the instrument, as well as the condition of hardware such as the power supply.  Allowing the instrument to warm up will result in less need to run the tune procedure and overall longer tuning stability. It may be necessary to retune the synth during operation. You should trust your ear on this.

\textbf{Per note tuning}

The automated tuning procedure described above applies 12 note equal tempered tuning. To achieve other tunings each of the 12 notes can be tuned manually. To enter the per note tuning press the \tune button in \shiftmode or \shiftlock. The \tune button flashes. In this mode the last played note (on any octave) can be tuned using the mod wheel. From the middle position the tune range is $\pm 1/2$ semitone from equal tempered. To leave the per note tuning mode press \tune again.

The per note tuning is stored as part of the patch (in contrast to the overall tuning, which is stored as part of the settings and applies to all presets) and it is retained after full tuning. However, the treatment is different for the active patch in \livemode. The Prophet-600 stores all additional patch parameters of the active patch in \livemode such that they can be recalled when \livemode is (re-)entered. The patch parameters include the per note tuning data. However, when the active patch parameters is recalled upon activation of the \livemode, the per note tuning is \underline{reset to equal tempered tuning}. So, when you edit the per note tuning for a patch you must stored it in a page in order to keep it.

The Prophet 600 is a VCO synthesizer, e.g. the frequency of the oscillators A and B are set by apllying a voltage to the control pin of the microchip. The relationship between absolute voltage and resulting frequency is not precise. It depends on many factors, such as smallest differences in the manufacturing process of the chip itself and the surrounding circuit components and operating temparture (the temparture of the components also changes during operation). Therefore the synthesizer need to be tuned regularly. Typically, tuning is best after the instrument (or rather: the chips and components) has been given time to warm up to a stable operating temperature, which could be 10-30 minutes into operation. The warm up time can depend on changes in air temperature, humidity and transportation of the instrument, as well as the condition of hardware such as the power supply. Therefore, after powering up, wait at least 5 to 20 minutes, and then, as needed, the TUNE button can be pressed to ensure optimal performance. It may be necessary to retune the synth during operation. The user should trust her or his ear on this. Allowing the instrument to warm up will result in less need to run the tune procedure and overall longer tuning stability.

\textbf{Per note tuning}

The automated tuning procedure applies 12 note equal tempered tuning. For other tunings each of the 12 notes can be tuned manually. The per note tuning is stored as part of the patch (in contrast to the overall tuning, which is stored as part of the settings and applies to all presets) and it is retained after full tuning. To enter the per note tuning press Tune in shift or shift-lock mode. The Tune button flashes. In this mode the last played note (on any octave) can be tuned using the mod wheel. From the middle position the tune range is $\pm$1 semitone from equal tempered. To leave the per note tuning mode press Tune again.

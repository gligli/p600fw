Each of the six voices of the Prophet-600 has two voltage controlled oscillators (VCOs). VCO A and VCO B have dedicated sub-panels. The oscillators are basically identical. They offer three waveforms: \textit{sawtooth}, \textit{triangle} and \textit{pulse}. The pulse width of the pulse shape can be adjusted using the corresponding \pulsewidth dial. All three shapes can be activated simultaneously. The figures below are for \textit{GliGli} panel layout and \textit{SCI} panel layout, which differ in the way you control the oscillator volumes.

\textbf{Oscillators in GliGli panel layout}

\begin{center}
\scalebox{0.4}{
  \begin{tikzpicture}[scale=0.8]
    \addpar{-18cm,13cm}{\oscpitchmode \\ This menu parameter oscillator sets the units of the two oscillator frequency dials, e.g. \textit{free}, \textit{semi}, \textit{octave}, or mixed (different units for A and B.) };
    \oscapanelsci{0cm,9.5cm}
    \oscbpanelsci{0cm,0cm}
    \mixerpanel{31.9cm,0cm}
    \draw[line width = 2pt, dashed] (-4cm,15cm) -- (1cm,15cm);
    \draw[line width = 2pt, dashed] (-4cm,15cm) -- (1cm,6cm);
  \end{tikzpicture}
}
\end{center}

\textbf{Oscillators in SCI panel layout}

\begin{center}
\scalebox{0.4}{
  \begin{tikzpicture}[scale=0.8]
    \addpar{-18cm,13cm}{\oscpitchmode \\ This menu parameter sets the units of the two oscillator frequency dials, e.g. \textit{free}, \textit{semi}, \textit{octave}, or mixed (different units for A and B.) };
    \draw[line width = 2pt, dashed] (-4cm,15cm) -- (1cm,15cm);
    \draw[line width = 2pt, dashed] (-4cm,15cm) -- (1cm,6cm);
    \oscapanelsci{0cm,9.5cm}
    \oscbpanelsci{0cm,0cm}
    
    \prophetpotbiplar{35cm, 14.5cm}{MIX}{75}
    \node[rectangle, font=\fontsize{14}{12}\selectfont, anchor = center] at (33cm,12.5cm) {\panelfont{OSC A}};
    \node[rectangle, font=\fontsize{14}{12}\selectfont, anchor = center] at (37cm,12.5cm) {\panelfont{OSC B}};

    \addpar{42cm,13cm}{\drive \\ This menu parameter modifies the drive factor of the oscillator mix dial (applies to SCI panel layout only)};
    \draw[line width = 2pt, dashed] (41cm,15cm) -- (38cm,15cm);
  
  \end{tikzpicture}
}
\end{center}

\textbf{Oscillator controls}

Both oscillators have a \oscfreq dial which determines the frequency range of the oscillators when played from the keyboard or via MIDI.  You can choose the granularity of the frequency dials: 
\begin{itemize}
  \item \textit{free}: the base frequency can be chosen freely. The dial value selection is continuous and the \display shows approximate values between 0 and 99 in the normal way
  \item \textit{semitone}: the base frequency can be selected in semitones. The dial value selection is discrete and the \display shows values from 0 to 63 (semitones).   
  \item \textit{octave}: the base frequency can be selected in octaves. The dial value selection is discrete and the \display shows values c0, c1, c2, c3, c4 or c5. 
\end{itemize}  
In addition there are mixed settings, e.g. \textit{octave-semitone}, \textit{semitone-octave}, \textit{octave-free} and \textit{free-octave} providing different granularity for oscillators A and B. The parameter to select the granularity is the additional patch parameter \oscpitchmode. 

The additional \freqfine dial for oscillator B is bipolar and provides frequency fine tuning with a range of about $\pm 1$ semitone. The main purpose of separate frequency settings per oscillator is to be able to detune them against one another. This can be used to create thick and lively patches (when the pitch difference is small), to create complex timbres with harmonics (when pitch difference is large, typically octaves) or even create intervals (for example fifths). The frequency setting is also an important aspect of poly-mod, see section \ref{polymod}. To be able to tune the relative frequency very accurately, \freqfine has a very sensitive (e.g. slow changing) mid region between -1 and 1.

The oscillator A has an additional \oscsync switch. The hardware of the Prophet-600 offers hard sync of oscillator A to oscillator B, e.g. when \oscsync is activated, oscillator A is reset to the cycle start when oscillator B completes a cycle. 

Finally, the outputs of both oscillators are mixed into one signal before entering the filter and output stages. The volume of oscillators A and B are either set by the \vola and \volb dials in GliGli panel layout or the \mixer and \drive parameters in SCI panel layout as described in the following. For details on switching the panel layout, see section \ref{panelswitch}.

\textbf{Oscillator mix in GliGli panel layout}

In GliGli panel layout the volume of each oscillator has a dedicate control in the mixer sub-panel. The normalised output of either oscillator is about half way of the respective dial. Beyond this, the amplifiers into the filter can be over-driven. This is a new feature of the upgraded firmware from version 2.0 on.

\textbf{Oscillator mix in SCI panel layout}

In the \textit{SCI} panel layout the two dials are designated \mixer (becoming a bipolar dial) and \glidepot and these control the oscillator mix and the glide time, respectively. For glide amount see section \ref{glide}. In order to cover the entire range of oscillator volumes which the GliGli controls provide, there is an additional menu parameter \drive which becomes available when the SCI panel layout is selected. The original Sequential Circuits mix function corresponds to a half-way value for \drive at which oscillators A and B are mixed linearly. At \mixer =-50 oscillator A is at full volume and oscillator B is silent. At \mixer =+50 oscillator B is at full volume and oscillator A is silent.  For smaller and larger values of \drive the oscillator mix function is modified as shown below.


\scalebox{0.4}{
  \begin{tikzpicture}[scale=0.8]
    \addpar{-2cm,14cm}{\textbf{Oscillator mix function} \\ In SCI panel layout the volumes of oscillator A and B are set by using the \mixer control and the menu parameter \drive.};

    \coordinate(orig)at(20cm,0cm);
      
      % A-axis
    \draw[-to, line width = 2pt] (orig) -- ($(orig)+(12cm,0cm)$); 
    \node[rectangle, font=\fontsize{17}{12}\selectfont, anchor = west] at ($(orig)+(12.2cm,0cm)$) {Osc B}; 
    \node[rectangle, font=\fontsize{15}{12}\selectfont, anchor = north] at ($(orig)+(10cm,-0.2cm)$) {Full}; 
    \node[rectangle, font=\fontsize{15}{12}\selectfont, anchor = north] at ($(orig)+(5cm,-0.2cm)$) {Half}; 
    \draw[-to, line width = 2pt] (orig) -- ($(orig)+(0cm,12cm)$); 
    \node[rectangle, font=\fontsize{17}{12}\selectfont, anchor = south] at ($(orig)+(0cm,12.2cm)$) {Osc A}; 
    \node[rectangle, font=\fontsize{15}{12}\selectfont, anchor = east] at ($(orig)+(-0.2cm,10cm)$) {Full}; 
    \node[rectangle, font=\fontsize{15}{12}\selectfont, anchor = east] at ($(orig)+(-0.2cm,5cm)$) {Half}; 

    \prophetpotbiplar{37cm,0cm}{MIXER}{310};
    \prophetpotbiplar{20cm,17cm}{MIXER}{240};
    \prophetpotbiplar{32.5cm,12.5cm}{MIXER}{90};

    \draw[to-to, line width = 2pt] ($(orig)+(10cm,0.2cm)$) -- ($(orig)+(0.2cm,10cm)$); 
    \node[rectangle, font=\fontsize{14}{12}\selectfont, rotate=-45, anchor = south] at ($(orig)+(4.2cm,6.2cm)$) {drive=50}; 

    \draw[to-to, line width = 2pt] ($(orig)+(5cm,0.2cm)$) -- ($(orig)+(0.2cm,5cm)$); 
    \node[rectangle, font=\fontsize{14}{12}\selectfont, rotate=-45, anchor = south] at ($(orig)+(2.4cm,3cm)$) {drive=25}; 

    \draw[to-to, line width = 2pt] ($(orig)+(2cm,0.2cm)$) -- ($(orig)+(0.2cm,2cm)$); 

    \draw[-to, line width = 2pt] ($(orig)+(7.6cm,7.6cm)$) -- ($(orig)+(0.6cm,9.9cm)$); 
    \draw[-to, line width = 2pt] ($(orig)+(7.6cm,7.6cm)$) -- ($(orig)+(9.9cm,0.6cm)$); 
    \node[rectangle, font=\fontsize{14}{12}\selectfont, rotate=-18.4, anchor = east] at ($(orig)+(7.7cm,8.04cm)$) {drive=75}; 

    \draw[-to, line width = 2pt] ($(orig)+(9cm,9cm)$) -- ($(orig)+(0.2cm,10.2cm)$); 
    \draw[-to, line width = 2pt] ($(orig)+(9cm,9cm)$) -- ($(orig)+(10.2cm,0.2cm)$); 
    \node[rectangle, font=\fontsize{14}{12}\selectfont, rotate=-6.39, anchor = east] at ($(orig)+(8.9cm,9.56cm)$) {drive=90}; 

  \end{tikzpicture}
}

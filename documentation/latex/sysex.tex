The firmware SysEx has the following structure:

\begin{itemize}
  \item Chain of data SysEx block, each representing a storage page in the Teensy++ 2.0
  \subitem "F0": SysEx start
  \subitem Prophet 600 signature: "00 61 16"
  \subitem Prophet 600 firmware message type: "6B"
  \subitem Pagesize (constant: "02 00" = 512 symbols)
  \subitem Page ID: one byte starting from highest ID
  \subitem 64 sets of 5 MIDI bytes (14 bit) (5 MIDI bytes are converted to 4 full bytes, this a total of 256 bytes for that page).
  \subitem 3 bytes of checksum
  \subitem "F7": SysEx end
  \item One termination SysEx
  \subitem "F0": SysEx start
  \subitem Prophet 600 signature: "00 61 16"
  \subitem Prophet 600 firmware message type: "6B"
  \subitem "00 00", instead of page size, signifying the end of the data stream
  \subitem "F7": SysEx end  
\end{itemize}


\textbf{Conversion of MIDI bytes to full bytes}

Each MIDI byte can hold values up to 127 (14 bit). To encode full bytes, 4 full bytes are stored in 5 MIDI bytes as follows. The first 4 MIDI bytes contain the first 7 bits of the 4 bytes to be encoded. The 8th bit of the 4 bytes is encoded in the first 4 bits of the 5th MIDI byte.   

Examples:

\begin{itemize}
  \item "01 00 00 00" is encoded in SysEx MIDI as "01 00 00 00 00"
  \item "80 00 00 00" is encoded in SysEx MIDI as "00 00 00 00 08"
  \item "FF 00 00 00" is encoded in SysEx MIDI as "7F 00 00 00 08"
  \item "00 00 00 FF" is encoded in SysEx MIDI as "00 00 00 7F 01"
  \item "FF FF FF FF" is encoded in SysEx MIDI as "7F 7F 7F 7F 0F"
\end{itemize}

\textbf{Checksum}

(this needs to be done)

The Prophet 600 comes with a (pitch) \textbf{Bender} (without automatic return to the middle position). With the upgrade it is now possible to customize the range and the target of the bender as part of a patch.

\begin{itemize}
  \item The bender can target the VCO frequency or the filter cut-off frequency. This is set using the addtional patch parameter \textbf{(6) Bend Target} which provides the options \textit{off}, \textit{VCO} (modulation of the frequency of both oscillators), \textit{VCF} (modulating the cut-off frequency) and \textit{amplitude} (modulating the amplitude).
  \item The range of the bender can be set using the additional patch parameter \textbf{(66) Bender Range}. It offers the choices \textit{2nd} (bend a full tone), \textit{3rd} (bend 3 semi-tones), \textit{5th} (bend 5 semi-tones) and \textit{octave} (bending an entire octave).
\end{itemize}

Note that the external MIDI pitch bend is added to the internal pitch bend. Full internal and external (MIDI) bend covers twice the range set by the bender range.

The upgraded Prophet 600 also offers he possibility to calibrate the bender middle position. This function can be used to remove jittery behaviour of the bender around the middle position. In order to calibrate the bender follow the steps below.

\begin{enumerate}
  \item Move the bender to the middle position.  
  \item While holding down the button From Tape push number "3" once (the display will show that the bender calibration mode is active) and then, without releaseing the From Tape button, press 3 again. The display indicates that the Pitch Wheel has been calibrated. 
\end{enumerate}

The bender calibration is not part of the patch but rather belongs to the group of miscellaneous settings, see section \ref{settingsref}. 

In preset modes or live mode the current sound can be stored by pressing the \textbf{Record} button, whose LED starts blinking. The Prophet 600 is then waiting for the user to enter the (two-digit) target patch number on the \textbf{Number Pad}. Once the second digit is entered the patch is stored automatically and the LED of the Record button deactivates. Once selected there is a short confirmation on the display.  

Loading patches is done by entering the preset patch mode and then selecting the patch on the number pad. Once selected there is a short confirmation on the display.

In patch number receive mode (record mode or preset patch mode) other functions of the number pad and display (e.g. in patch panel mode or live mode) are temporarily suppressed. If you find you have accidentally started typing a digit in patch number receive mode while expecting to select a patch parameter you can switch / cancel the mode after the first digit, e.g. pressing To Tape in preset patch mode or pressing Record again in record mode.

For exporting and importing patches via MIDI SysEx see section \ref{mididump}.

Patches are stored in \storagemode. To enter this mode from \presetmode or \livemode press the \record button, whose LED starts blinking. The Prophet 600 is then waiting for the user to enter the (two-digit) target patch number on the \termnumberpad. Once the second digit is entered, the patch is stored automatically and the LED of the \record button deactivates. 

Loading patches is done by entering \presetpatch and then selecting the patch on the \termnumberpad. Once selected there is a short confirmation on the display.

While the Prophet 600 waits for patch number entry (e.g. \storagemode or \presetpatch) other functions of the \termnumberpad and display (e.g. in \presetpanel or \livemode) are temporarily suppressed. If you find you have accidentally started typing a digit in these modes while expecting to select a patch parameter you can switch / cancel the mode after the first digit, e.g. pressing \totape in \presetpatch or pressing \record again in \storagemode.

For exporting and importing patches via MIDI SysEx see section \ref{patchmgmt}.

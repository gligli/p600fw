Patches are stored in \storagemode. To enter this mode from \presetmode or \livemode press the \record button, whose LED starts blinking. The Prophet 600 is then waiting for you to enter the (two-digit) target patch number on the \termnumberpad. Once the second digit is entered, the patch is stored automatically and the LED of the \record button deactivates. 

Loading patches is done in \presetpatch by selecting the patch on the \termnumberpad. Once selected and loaded the display then shows the selected patch number. If there is no valid patch stored at the selected page then no new patch is loaded and the Prophet 600 remains at the active patch. You can also re-load the current active patch, for example if you changed something and you would like to go back to the original. 

For even easier and faster access the patch selection can also be done using the \datadial. If you are in \presetmode and hold \fromtape then \datadial selects the patch. Notice that in contrast to the patch selection using the number pad, the page selected using the \datadial will always "load" even if there is no patch stored there. This means that the display shows the selected page number in any case, but the instrument remains silent if there is no patch stored there. To be precise, the Prophet 600 loads the default patch with no wave form activated in this case. 

While the Prophet 600 waits for patch number entry (e.g. \storagemode or \presetpatch) other functions of the \termnumberpad and display (e.g. in \presetpanel or \livemode) are temporarily suppressed. If you find you have accidentally started typing a digit in these modes while expecting to select a patch parameter you can switch / cancel the mode after the first digit, e.g. pressing \totape in \presetpatch or pressing \record again in \storagemode.

For exporting and importing patches via MIDI SysEx see section \ref{patchmgmt}.

As the original model, the upgraded Prophet 600 offers a single LFO (apart from vibrato, described in scetion \ref{vib}). However, the functionality is substantially enhanced. The frequency range of the LFO is much wider, from 1/20 to 60 Hz. There are additional wave forms (a total of 6). The upgraded Prophet 600 offers more modulation target options and, finally, an LFO delay can be set. These enhancements require 4 additional patch parameters as described in the following. However, apart from these parameters the most important LFO controls can be found and be changed hands-on in the dedicated LFO panel section. 

\textbf{LFO speed}

The LFO speed is set in the panel using the \textbf{LFO Frequency} dial. In order to provide fine tuning on a broader frequency there are two basic frequency ranges, \textit{slow} and \textit{fast}. This is toggled using an additional patch parameter \textbf{(22) LFO Range}. 

\textbf{LFO modulation amount and targets}

There are three basic modulation targets which can be active simultaneously. Each one is activated using the switches \textbf{VCO Frequency}, \textbf{Pulse Width} and \textbf{Filter Frequency} in the LFO panel section. The modulation amount is set using the \textbf{Initial Amount} dial and this amount applies to all activated modulation targets. 

In the upgrade Prophet 600 the target VCO Frequency can be cutomized in the following way. The user can choose which of the two oscilliators per voice is modulation. The additional patch parameter \textbf{(11) LFO Target} offer the settings: \textit{VCO A \& B} (both osciallators are modulated, this is the standard of the original Prophet 600), \textit{VCO A} (only VCO A ist modulated), \textit{VCO B} (only VCO B ist modulated),  and \textit{VCO A \& B \& VCA} (both osciallators and amplitude is modulated). The targets pulse width and filter frequency are fixed, e.g. they can only be switch on or off and there are no further parameters for customization.

\textbf{LFO shapes}

The upgraded Prophet 600 supports not only the waveform \textit{Pulse} and \textit{Triangle} of the original model but four additional shapes, \textit{Sin}, \textit{Sawtooth}, \textit{Random} and \textit{Noise}. On the panel the is a simple up/down \textbf{Shape} switch. To accommodate 6 waveforms an additional patch parameter \textbf{(1) LFO Shape} is used. Selecting the desired waveform therefore requires selecting the respective shape pair and the using the \textbf{Shape} switch to change between the two shapes. 

\begin{tabular}{ |l|l|l|} 
 \hline
  \textbf{(1) LFO Shape} set to: & \textbf{LFO} switch \textit{up} & \textbf{LFO} switch \textit{down} \\
 \hline
 \textit{tri-square} & LFO shape is triangle & LFO shape is square\\
 \hline
 \textit{sin-rand} & LFO shape is sin & LFO shape is random \\
 \hline
 \textit{saw-noise} & LFO shape is sawtooth & LFO shape is noise \\
 \hline
\end{tabular}

\textbf{LFO delay and modulation of LFO}

As part of the patch parameters, the strength LFO effect itself can be modulated by either the modulation wheel or a delay function. The additional patch parameter \textbf{(44) Modulation Wheel Target} can be set to either \textit{vibrato} or \textit{LFO} (or \textit{off}. When set to \textit{LFO} the modulation wheel controls the LFO amount, e.g. the LFO starts at initial amount as set using the \textbf{Initial Amount} dial on the panel and turning the modulation wheel up adds LFO strength. The wheel effect itself is set using using the additional patch parameter \textbf{(33) Modulation Wheel Range}. If the modulation wheel target is set to \textit{vibrato} then, automatically, the the modulation delay is applied to the LFO, e.g. the LFO onset is delayed by the amount set by the additional patch parameter \textbf{(4) Modulation Delay}. For a details on moudlation wheel related parameters see section \ref{modwheel}.

\textbf{LFO Sync}

As a new feature in version 2.2 the LFO can be sync'ed to the arpeggiator / seequencer. The additional patch parameter \textbf{(111) LFO Sync} provides choices \textit{off} (free running LFO) and \textit{1, 2, 3, 4, 5, 6, 8} to have the LFO reset after 1, 2, 3, 4, 5, 6 or 8 beats. 

As with the original Prophet 600 firmware, there are two modes of operation, a \textbf{Preset} mode and a \textbf{Live} mode. The user can change between the two by pressing the \textbf{Preset} button. 

The idea of the live mode is that the patch that is playing corresponds to exactly the parameters as currently set by all dials, switches and additional patch parameters. The synth produces the sound as set and seen. In the preset mode, in contrast, the patch parameter as stored in the patch memory are applied. The synth produces the the sound of the preset. Still, once a dial or a switch on the panel or an additional parameter is changed, this change will take effect (note that this may lead to discontinuous changes if for example the current position of a knob is far from the  value stored in the patch). In this case  "mixed state" applies, partly patch (untouched controls) and partly panel controls (touched controls). The user can then store the current sound in the same patch or a different patch. The patch will be stored as heard. 

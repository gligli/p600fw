For performing the upgrade on a Prophet 600 with an original Z80 processor please refer to the GliGli modification instructions \cite{modinstructions}. The following section is for owners of a modified Prophet 600 with Teensy board and a prior version of the GliGli upgrade installed. If modifying a Prophet 600 for the time one may directly flash the Teensy board with software version \version. It is not necessary to first install a lesser version and then upgrade.

Firmware upgrades are provided as MIDI SysEx files (*.syx) and as Teensy code files (*.hex) for USB transfer to the Teensy board. 

\textbf{Firmware upgrade via SysEx}

In order to perform the upgrade follow to steps below.

\begin{enumerate}
  \setlength\itemsep{0cm}
  \item Store and save the patch bank and make a note of important settings
  \item Switch off the instrument
  \item Switch on the instrument while holding the buttons \totape and \fromtape down. The display should show an "U" for upgrade on the left digit.
  \item Send the firmware SysEx file to the instrument. This process takes a while. The display shows a spinning segment. Once the process is completed, the display shows an "S" for success. If the upgrade ends with “E” for error on the display, the upgrade has failed and the upgrade procedure must be repeated until it succeeds. It is not recommended to start the instrument in normal mode after a failed upgrade.
  \item Switch the instrument off and on again. The scrolled welcome message should state the firmware version 
\end{enumerate}

Users are recommended to follow the MIDI SysEx part of the modification instructions \cite{modinstructions}. In particular, the "Delay between buffers" and the "delay after F7" should be set to at least 250ms.

The settings and patch parameters are left unchanged by the upgrade, provided the previously installed version is one for which backward compatibility is supported. This applies to version 2.0 and 2.1 RC3 3. Downgrading versions is possible but preservation of settings and preset parameters is not guaranteed in that case.

\textbf{Firmware upgrade via USB}

Loading the firmware code to the Teensy board via USB has advantages but several aspects need to be taken into consideration. Here are some of the most important differences.

\begin{itemize}
  \item While upgrade by SysEx works via the MIDI In socket, the USB transfer requires opening up the Prophet 600
  \item USB transfer requires downloading and installing a pjrc software on a computer    
  \item USB transfer erases all patch and settings data including tuning
  \item USB transfer is very fast (seconds) and 100\% reliable
  \item Unless the Teensy board is temporarily removed, the USB transfer requires access to the Teensy board from the side which is blocked by the storage battery of the Prophet
  \item Unless the Teensy board is temporarily removed, the transfer has to be done with power switched on and with the Prophet 600 panel lid open in at least one task 
\end{itemize}

To reach the Teensy board unscrew the top 2 screws on the wooden side panels on either either. After this, the panel can be lifted up from top end of the keyboard. It is advisable to be careful when opening the panel lid because there are delicate wire connection from the keyboard and the circuit boards inside the Prophet 600 and the circuits board attached to the panel lid. These connection can break, in particular when they have become brittle with age. The is no real danger of serious damager but if broken, the wires need to be re-soldered.

\framebox[19cm]{
  \begin{minipage}{18cm}
      \textbf{Attention: the following instructions involve opening up the Prophet 600 and preforming little tasks inside unit while the power is on. Obviously, there is the danger of electrical shock. You do this at your own risk and you must consult a professional in case of doubt.}  
  \end{minipage}
 }

The most convenient method upgrading the Teensy code via USB is to use the power supplied by the Prophet 600 when it is in the socket. In this case the procedure neither requires that the Teensy is taken out of the Prophet 600 and it is then not required that the 5V bridge in the Teensy modification be reconnected (see GliGli modification instructions \cite{modinstructions}). 

\begin{enumerate}
  \item Download the *.hex file for the firmware version you want to install 
  \item With the panel lid open and the power off connect a USB cable to the USB socket of the Teensy on the side which faces the rear side of the unit. You need a USB cable with a type A plug for this. 
  \item Connect the cable to your computer 
  \item Download an install the Teensy software from pjrc \cite{teensyloader} and open the application 
  \item Power up the Prophet 600 and then press the button on the Teensy board to activate code transfer mode. Follow the instructions provided with the Teensy loader software. 
  \item When the code transfer mode is active, select the *.hex file provided for the release version you want to install and click \textit{program}
  \item After the code has been transferred click \textit{reboot}. When you close the panel lid you will see the Prophet 600 entering tuning mode. This is necessary and automatic since the settings (including tuning data ) have been wiped from the Teensy storage. 
  \item After the tuning procedure is complete, the instrument is ready to play. All patch slots will be empty and the active patch in preset mode is the default patch. 
\end{enumerate}

One practical problem is that the USB socket of the Teensy board in place is blocked by the battery. The battery is only needed to store patches when the Prophet 600 is operated with the Z80 in place. With the Teensy upgrade it is no longer needed. To solve the problem there are three possibilities: 1) remove the battery, 2) stack several 2x20 sockets on top of each other so that the Teensy board / the USB socket is above the battery or 3) implement a different solution for the battery for example a battery holder attached at a convenient place inside the Prophet and connected with wires to the proper $\pm$ connections. 

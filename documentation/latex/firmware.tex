The Teensy++ 2.0 hardware uses a memory which is structured in pages of 256 bytes.

The firmware SysEx has the following structure:

\begin{itemize}
  \item Chain of data SysEx block, each representing a storage page in the Prophet 600
  \subitem "F0": SysEx start
  \subitem Prophet 600 signature: "00 61 16"
  \subitem Prophet 600 firmware message type: "6B"
  \subitem Pagesize (constant: "02 00" (= 512)
  \subitem Page ID: one byte starting from highest ID
  \subitem 64 sets of 5 MIDI bytes (14 bit) converted into 4 bytes. This makes 256 bytes.
  \subitem 3 bytes of checksum
  \subitem "F7": SysEx end
  \item One termination SysEx
  \subitem "F0": SysEx start
  \subitem Prophet 600 signature: "00 61 16"
  \subitem Prophet 600 firmware message type: "6B"
  \subitem "00 00", instead of page size, signifying the end of the data stream
  \subitem "F7": SysEx end  
\end{itemize}

The data is written into application part of the storage, into the RWW (read-while-write) section of the memory. The number of pages used a firmward version can be obtained by counting the pages contained in the SysEx file.

Patch data is also written into the RWW memory. The patch 0 starts at page pointer 256. The sequencer data is stored in two pages starting at 456. (456 for track 1 and 457 for track 2). The interim storage of the preset page is located on page pointer 475 and the settings are stored in 2 pages starting at pointer 476.

Note to developers: the application code must fit into the first 256 pages. This corresponds to a SysEx size of 85258 kB which is the equivalent of 65536 kB hex file. Code beayon this size will overwrite patches and will be corrupted by patche storage operations.


The upgraded Prohpet 600 has two methods for controlling the modulation effect of the LFO and the vibrato and these properties are  stored with the patch. The first one is the \textbf{Modulation Wheel} which can be found in the performance section. The target of the modulation wheel can be set using the additional patch parameter \textbf{(44) Modulation Wheel Target} which provides the choices \textit{LFO} or \textit{vibrato} (or \textit{off}). The second method is a modulation delay which delays the onset of the modulation effect. The delay is always applied to the modulation source which is \underline{not controlled} by the modulation wheel, e.g. when the modulation wheel target is set to \textit{LFO} the delay applies to the vibrato and when it is set to \textit{vibrato} the delay is applied to the LFO.

The target setting as well as the modulation wheel strength and the delay time are set as follows. 
\begin{itemize}
  \item \textbf{(4) Modulation Delay} is parameter 4, numeric setting 0...98
  \item \textbf{(44) Modulation Wheel Target} is parameter 44, choices are: \textit{vibrato} or \textit{LFO}
  \item \textbf{(33) Modulation Range} is parameter 33. The, choices are: \textit{min}, \textit{low}, \textit{high} and \textit{full}
\end{itemize} 

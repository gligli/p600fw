The upgraded Prophet-600 has two methods for controlling the modulation effect of the LFO and the vibrato and these properties are  stored with the patch. The first one is the \modwheel. The target of the modulation wheel can be set using the menu patch parameter \modwheeltarget which provides the options \textit{LFO} and \textit{vibrato}. 

The second method is a modulation delay which delays the onset of the modulation effect. The delay is always applied to the modulation source which is \underline{not controlled} by the \modwheel. So when the modulation wheel target is set to \textit{LFO} the delay applies to the vibrato and when it is set to \textit{vibrato} the delay is applied to the LFO.

The target setting as well as the modulation wheel strength and the delay time are set as follows. 
\begin{itemize}
  \item \modwheeltarget provides choices \textit{LFO} and \textit{vibrato}
  \item \moddelay is a continuous menu parameter (range 0...99) which controls the delay time
  \item \modwheelrange offers 4 options: \textit{touch}, \textit{soft} (this is the default), \textit{high} and \textit{full}
\end{itemize} 

The parameter \modwheelrange affects the \modwheel action for all variations except \textit{VCA}. If the target is \textit{vibrato} and the vibrato target is \textit{VCA} the wheel action is always effectively \textit{full}. The reason is that amplitude modulation is a weak effect and there is no practical value in making the \modwheel softer in this case.

Note: the \modwheel action has been modified in all settings compared to version 2.0 and 2.1 RC3. It is now exponential to provide a smooth onset for better playability.

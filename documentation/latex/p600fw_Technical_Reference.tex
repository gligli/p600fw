\documentclass[landscape, 11pt, oneside]{report}
\usepackage[a4paper, left=2cm, right=2cm, top=2.5cm]{geometry}
\usepackage{lmodern}
\usepackage{hyperref}
\usepackage{titlesec}
\usepackage{scrextend}
\usepackage{array}
\usepackage{ProphetPanel}
\usepackage{fontspec}
\usepackage[T1]{fontenc}
\usepackage{longtable}
\usepackage{ProphetTerms}
\usepackage{float}
%\usepackage[inline]{showlabels}
\setmainfont[
 BoldFont={FiraSans-Bold}, 
 ItalicFont={FiraSans-Italic},
 ]{FiraSans}
\setlength{\parindent}{0pt}
\setlength{\parskip}{1em}
\setlength{\leftmargin}{-4cm}
\setlength{\paperheight}{210mm}
\setlength{\paperwidth}{297mm}
\setlength{\textwidth}{26cm}
\setlength{\footskip}{2.5cm}
%% globals
\newcommand{\version}{2.X}
\newcommand{\paramgeneral}[1]{\textbf{#1}}
\newcommand{\button}[1]{\textbf{#1}}
\newcommand{\patchparameter}[2]{\paramgeneral{#2 - #1}}

% additional patch parameters
\newcommand{\clockspeed}{\patchparameter{0}{Clock/Speed}}
\newcommand{\lfoshape}{\patchparameter{1}{LFO Shape}}
\newcommand{\lfotarget}{\patchparameter{11}{LFO Target}}
\newcommand{\lfosync}{\patchparameter{111}{LFO Sync}}
\newcommand{\vibspeed}{\patchparameter{2}{Vibrato Speed}}
\newcommand{\lforange}{\patchparameter{22}{LFO Frequency Range}}
\newcommand{\vibamt}{\patchparameter{3}{Vibrato Amount}}
\newcommand{\modwheelaction}{\patchparameter{33}{Modulation Wheel Action}}
\newcommand{\moddelay}{\patchparameter{4}{Modulation Delay}}
\newcommand{\modwheeltarget}{\patchparameter{44}{Modulation Wheel Target}}
\newcommand{\2ndenv}{\patchparameter{5}{2nd Envelope Shape}}
\newcommand{\filenv}{\patchparameter{55}{Filter Envelope Shape}}
\newcommand{\envrouting}{\patchparameter{555}{Envelope Routing}}
\newcommand{\bentarget}{\patchparameter{6}{Bender Target}}
\newcommand{\bendrange}{\patchparameter{66}{Bender Target}}
\newcommand{\glide}{\patchparameter{7}{Glide Time}}
\newcommand{\assignprio}{\patchparameter{77}{Assigner Priority}}
\newcommand{\detune}{\patchparameter{8}{Unison Detune Amount}}
\newcommand{\oscpitchmode}{\patchparameter{88}{Oscillator Pitch Mode}}
\newcommand{\spread}{\patchparameter{888}{Spread Amount}}
\newcommand{\ampvel}{\patchparameter{9}{Amplitude Velocity}}
\newcommand{\filvel}{\patchparameter{99}{Filter Velocity}}
\newcommand{\syncbug}{\patchparameter{999}{Sync Bug Switch}}


% buttons
\newcommand{\numberpad}[1]{\textbf{button #1}}
\newcommand{\totape}{\button{To Tape}}
\newcommand{\fromtape}{\button{From Tape}}
\newcommand{\tune}{\button{Tune}}
\newcommand{\preset}{\button{Preset}}
\newcommand{\record}{\button{Record}}
\newcommand{\seq1}{\button{Seq 1}}
\newcommand{\seq2}{\button{Seq 2}}
\newcommand{\arpud}{\button{Arp U/D}}
\newcommand{\arpass}{\button{Arp Assign}}

% panel controls
\newcommand{\polymodenv}{\patchparameter{Poly-Mod}{Envelope}}
\newcommand{\polymodosc}{\patchparameter{Poly-Mod}{Oscillator B}}
\newcommand{\polymodfreq}{\patchparameter{Poly-Mod}{Frequency Target}}
\newcommand{\polymodfilter}{\patchparameter{Poly-Mod}{Filter Target}}

\newenvironment{flowtext}{\addmargin[0cm]{7cm}}{\endaddmargin} % standard text width (reduced for layout)

\title{[DRAFT] Technical Reference for the GliGli upgraded Prophet 600}
\author{Florian Merz (Editor)}
\pagestyle{myheadings}\markright{[DRAFT]  Technical Reference for the GliGli upgraded Prophet 600}
\makeatletter
\renewcommand\chapter{\pagestyle{myheadings}\markright{[DRAFT]  Technical Reference for the GliGli upgraded Prophet 600}\global\@topnum\z@\@afterindentfalse\secdef\@chapter\@schapter}
\makeatother
\titleformat{\chapter}[display]{\pagebreak\LARGE\bfseries}{}{0.0cm}{}
\titlespacing{\chapter}{0pt}{*0}{*0}
%\titleformat{\section}[display]{\Large\bfseries}{}{0.0cm}{}
\titlespacing{\section}{0pt}{*0}{*0}

\begin{document}

%\maketitle
\tableofcontents


\pagebreak

\chapter{Overview of technical implementation}

\begin{flowtext}
The Teensy++ 2.0 hardware uses a memory which is structured in 512 pages with 256 bytes each.

(this needs to be done)
\end{flowtext}

\chapter{Implementation}

\begin{flowtext}

\section{MIDI bytes to full byte conversion}\label{midibyteconversion}

The data section in Prophet 600 MIDI SysEx commands consist of blocks of 5 MIDI bytes. Each MIDI byte can hold values up to 127 (14 bit). 4 full bytes are stored in 5 MIDI bytes as follows. The first 4 MIDI bytes contain the first 7 bits of the 4 bytes to be encoded. The 8th bit of the 4 bytes is encoded in the first 4 bits of the 5th MIDI byte.   

Examples:

\begin{itemize}
  \item "01 00 00 00" is encoded in SysEx MIDI as "01 00 00 00 00"
  \item "80 00 00 00" is encoded in SysEx MIDI as "00 00 00 00 08"
  \item "FF 00 00 00" is encoded in SysEx MIDI as "7F 00 00 00 08"
  \item "00 00 00 FF" is encoded in SysEx MIDI as "00 00 00 7F 01"
  \item "FF FF FF FF" is encoded in SysEx MIDI as "7F 7F 7F 7F 0F"
\end{itemize}
 
In this way, a sequence of $X$ MIDI blocks (5 MIDI bytes) is be converted to a bytes sequence of $4\dot X$ full bytes. Note, however, that double byte parameter (which is applies to most continuous patch parameters) are sequenced such that in the full bytes sequence the least significant byte (LSB) comes before the most significant byte (MSB), see section on patch data encoding in MIDI SysEx.  

\section{Patch SysEx structure}\label{sysex}

Patch parameters are stored as patches in each of the 100 slots on the Teensy++ 2.0 board (see section on memory usage). Apart from technical overhead this covers:
\begin{itemize}
  \item panel control settings
  \item menu parameters
  \item special data: per note tuning
  \item special data: note pattern (when the patch has unison activated with a chord pattern, this data contains the latched pattern) 
  \item special data: patch name 
\end{itemize}

The SysEx MIDI data is build directly from the storage data. It therefore has the same structure (order of parameters). 

\subsection{Storage version}

Due to extensions of the parameter scope for additional functions, the patch storage structure depends on the firmware version. The storage versioning is as follows:

\begin{itemize}
  \item Firmware version 2.0: storage version 2
  \item Firmware version 2.1 RC3: storage version 7
  \item Firmware version \version: storage version 8  
\end{itemize}

For the implementation of an MIDI SysEx based patch editor it is crucial to take storage version into account. It is not recommended to try to convert a lower version SysEx file to a higher version, as several parameters have the value ranges changed. Instead one should always import a lower version SysEx patch file to the Prophet-600 with a newer firmware installed that uses the storage version you want and then export it again. The exported SysEx patch file will be properly converted. 

Note: it would be possible to document the transformations / remappings between versions 7 and 8. If you feel you need this for what you are trying to do, please get in contact via https://github.com/image-et-son/p600fw.

\subsection{Patch MIDI SysEx structure}

The SysEx structure for patch data is as follows:

\begin{enumerate}
  \item "F0": SysEx start
  \item Prophet-600 SysEx command signature: "00 61 16"
  \item Prophet-600 SysEx patch dump type: "01"
  \item Patch data section: series of MIDI data blocks, each consisting of 5 MIDI bytes (14 bit) 
  \item "F7": SysEx end
\end{enumerate}

The Prophet-600 ignores SysEx commands which do not carry the signature "00 61 16" with the exception of original Z80 produced SysEx files.

The data part of the patch SysEx has the same underlying structure as the internal storage in the sense that the parameters are stored in exactly the same order. As described in section \ref{midibyteconversion} the storage of double  byte values is added to the MIDI byte sequence in the order \textit{LSB}, \textit{MSB}. E.g the lower byte comes first. This also applies if a double byte is split across two MIDI data blocks (e.g. the \textit{LSB} is the fourth byte of one MIDI data block and the \textit{MSB} is the first byte of the next MIDI data block). 

The technical overhead inside the patch data section (item 4 in the list above) consists of another specific signature (\textit{Storage Magic}) and the stroage version. The signature is used to identify whether the data section of the SysEx represents a valid storage page of the Prophet-600. It should  always be present for any valid patch data independently from the storage version. If that signature is not present or differs the patch data will be ignored. The storage version should be used to identify the correct mapping of values.  

Note that the MIDI SysEx patch data contains all parameters up to and including the last non-zero value. It therefore has variable a size. However, it always consists of complete MIDI blocks (5 MIDI bytes which convert into 4 data bytes).

Each patch SysEx command as shown above represents exactly one flash storage page on the Teensy++ 2.0 board. A patch bank dump simply consists of a sequence of closed SysEx patch commands.

\subsection{Patch storage structure}\label{patchstore}

The following table specifies the patch data storage up to and including storage version 8. 

\footnotesize
\renewcommand{\arraystretch}{1.3}

\begin{longtable}[l]{p{5cm}|p{2cm}|p{1.5cm}|p{1.5cm}|p{2cm}|p{2.2cm}|p{5cm}} 
\textbf{Patch Data} & \textbf{Type} & \textbf{Byte} & \textbf{Count} & \textbf{Total Bytes} & \textbf{From Version} & \textbf{Stepped value} \\ \hline
\endfirsthead
\textbf{Patch Data} & \textbf{Type} & \textbf{Byte} & \textbf{Count} & \textbf{Total Bytes} & \textbf{From Version} & \textbf{Stepped value} \\ \hline
\endhead 
Patch Number & Technical & 1 & 1 & 1 & 0 &  \\ \hline
Storage Key (aka MAGIC) & Technical & 4 & 1 & 4 & 0 &  \\ \hline
Storage Version & Technical & 1 & 1 & 1 & 0 &  \\ \hline
Fequency A & Continuous & 2 & 1 & 2 & 1 &  \\ \hline
Volume A & Continuous & 2 & 1 & 2 & 1 &  \\ \hline
PWA & Continuous & 2 & 1 & 2 & 1 &  \\ \hline
Fequency B & Continuous & 2 & 1 & 2 & 1 &  \\ \hline
Volume B & Continuous & 2 & 1 & 2 & 1 &  \\ \hline
PWB & Continuous & 2 & 1 & 2 & 1 &  \\ \hline
Frequency Fine B & Continuous & 2 & 1 & 2 & 1 &  \\ \hline
Cutoof & Continuous & 2 & 1 & 2 & 1 &  \\ \hline
Resonance & Continuous & 2 & 1 & 2 & 1 &  \\ \hline
Filter Envelope Amount & Continuous & 2 & 1 & 2 & 1 &  \\ \hline
Filter Release & Continuous & 2 & 1 & 2 & 1 &  \\ \hline
Filter Sustain & Continuous & 2 & 1 & 2 & 1 &  \\ \hline
Filter Decay & Continuous & 2 & 1 & 2 & 1 &  \\ \hline
Filter Attack & Continuous & 2 & 1 & 2 & 1 &  \\ \hline
Amp Release & Continuous & 2 & 1 & 2 & 1 &  \\ \hline
Amp Sustain & Continuous & 2 & 1 & 2 & 1 &  \\ \hline
Amp Decay & Continuous & 2 & 1 & 2 & 1 &  \\ \hline
Amp Attack & Continuous & 2 & 1 & 2 & 1 &  \\ \hline
Poly Mod Envelope Amount & Continuous & 2 & 1 & 2 & 1 &  \\ \hline
Poly Mod OSC B & Continuous & 2 & 1 & 2 & 1 &  \\ \hline
LFO Frequency & Continuous & 2 & 1 & 2 & 1 &  \\ \hline
LFO Amount & Continuous & 2 & 1 & 2 & 1 &  \\ \hline
Glide & Continuous & 2 & 1 & 2 & 1 &  \\ \hline
Amp Velocity & Continuous & 2 & 1 & 2 & 1 &  \\ \hline
Filter Velocity & Continuous & 2 & 1 & 2 & 1 &  \\ \hline
Saw A & Stepped & 1 & 1 & 1 & 1 & Off, On \\ \hline
Tri A & Stepped & 1 & 1 & 1 & 1 & Off, On \\ \hline
SQR A & Stepped & 1 & 1 & 1 & 1 & Off, On \\ \hline
Saw B & Stepped & 1 & 1 & 1 & 1 & Off, On \\ \hline
Tri B & Stepped & 1 & 1 & 1 & 1 & Off, On \\ \hline
SQR B & Stepped & 1 & 1 & 1 & 1 & Off, On \\ \hline
Sync & Stepped & 1 & 1 & 1 & 1 & Off, On \\ \hline
Poly Mod Frequency A & Stepped & 1 & 1 & 1 & 1 & Off, On \\ \hline
Poly Mod Filter & Stepped & 1 & 1 & 1 & 1 & Off, On \\ \hline
LFO Shape & Stepped & 1 & 1 & 1 & 1 & Pulse, Triangle, Random, Sin, Noise, Saw\footnote{In the patch SysEx the combined effect of the LFO shape panel switch and the additional patch parameter for further LFO shape options is combined into this single stepped parameter with 6 options. An external editor could therefore use a single 6-way parameter for the LFO shape.} \\ \hline
(unused) & n/a & 1 & 1 & 1 & n/a\footnote{This slot was used for \textit{LFO Frequency Range} from versions 1 to 7} &  \\ \hline
LFO Targets & Stepped & 1 & 1 & 1 & 1 & see section \ref{lfotarget}  \\ \hline
Tracking Shift & Stepped & 1 & 1 & 1 & 1 & Off, Half, Full \\ \hline
Filter Envelope Shape & Stepped & 1 & 1 & 1 & 1 & Linear, Exponential \\ \hline
Filter Envelope Speed & Stepped & 1 & 1 & 1 & 1 & Slow, Fast \\ \hline
Amp Envelope Shape & Stepped & 1 & 1 & 1 & 1 & Linear, Exponential \\ \hline
Amp Envelope Speed & Stepped & 1 & 1 & 1 & 1 & Slow, Fast \\ \hline
Unison & Stepped & 1 & 1 & 1 & 1 & Off, On \\ \hline
Assigner Priority & Stepped & 1 & 1 & 1 & 1 & Last, Low, High \\ \hline
Bender Semitones & Stepped & 1 & 1 & 1 & 1 & 2nd, 3rd, 5th, Oct\footnote{Note that the numerical values for this option range are: 2, 3, 5 and 8, i.e. value corresponds to the actual semitones rather than the index of the selected option.} \\ \hline
Bender Target & Stepped & 1 & 1 & 1 & 1 & Off, A \& B, VCF, Volume, B \\ \hline
Mod Wheel Shift & Stepped & 1 & 1 & 1 & 1 & Touch, Soft, High, Full \\ \hline
Chromatic Pitch & Stepped & 1 & 1 & 1 & 1 & Free, Semi,Octave, Octave-Semi, Semi-Octave, Octave-Free, Free-Octave \\ \hline
Modulation Delay & Continuous & 2 & 1 & 2 & 2 &  \\ \hline
Vibrato Frequency & Continuous & 2 & 1 & 2 & 2 &  \\ \hline
Vibrato Amount & Continuous & 2 & 1 & 2 & 2 &  \\ \hline
Unison Detune & Continuous & 2 & 1 & 2 & 2 &  \\ \hline
(unused) & n/a & 2 & 1 & 2 & n/a\footnote{This slot was used for arpeggiator/sequencer clock speed fro versions 1 to 7. From version 8 on, clock speed is consistently implemented as a setting only.} &  \\ \hline
Modulation Wheel Target & Stepped & 1 & 1 & 1 & 2 & LFO, Vibrato \\ \hline
Vibrato Target & Stepped & 1 & 1 & 1 & 2 & VCO A \& B, VCA, VCO A, VCO B \\ \hline
Voice Pattern (6 voices) & Special & 1 & 6 & 6 & 2 &  \\ \hline
Tuning per Note (12 notes) & Special & 2 & 12 & 24 & 7 &  \\ \hline
Sync Bug & Stepped & 1 & 1 & 1 & 8 & Off, On \\ \hline
Vintage (Spread) & Continuous & 2 & 1 & 2 & 8 &  \\ \hline
External CV & Continuous & 2 & 1 & 2 & 8 &  \\ \hline
Envelope Routing & Stepped & 1 & 1 & 1 & 8 & Standard, Poly-Amp, Poly, Gate \\ \hline
Voice Assign & Stepped & 1 & 1 & 1 & 8 & First, Cycle \\ \hline
LFO Sync & Stepped & 1 & 1 & 1 & 8 & Off, Key, 1, 2, 3, 4, 5, 6, 8 \\ \hline
Patch Name & Technical & 1 & 16 & 16 & 8 &  \\ \hline




\end{longtable}


\textbf{Example}

Consider building the SysEx patch files for patch number 5 from storage in the latest version 8. The first 8 data bytes prior to conversion are: "05 00 61 16 A5 08 00 FF", where 05 is the patch number, "00 61 16 A5" is the \textit{Storage Magic}, "08" is the storage version and "00 FF" is an example value for the first real patch parameter which is frequency of oscillator A. These 8 data bytes are first rearranged into two blocks of 4 bytes like: "05 A5 16 61" "00 08 FF 00". Note that the order of bytes for each data block is reversed with \textit{LSB} first, etc. Then the conversion to 5 MIDI bytes leads to two blocks of 5 MIDI bytes like this: "05 25 16 61 02" "00 08 7F 00 04".


\subsection{Patch Name}

Up to version \version the Prophet-600 does no display the patch name and offers no option the change the patch name. However, the patch name data is stored in the patch and is preserved when the patch is send to Prophet-600 via MIDI SysEx, then stored on the unit and then exported again in SysEx format. An external SysEx patch editor could therefore make use of the patch name.

The patch consists of 16 bytes and the enconding is UTF-8 with printable characters as follows:

\begin{itemize}
  \item 00: NULL character terminates the patch name
  \item 01 - 0F: non-printable characters are interpreted as \textit{space}
  \item 20 - 7E: standard UTF-8 printable charatcers from \textit{space} through number, letters and special characters up to \textasciitilde
  \item 7F - FF: DEL and following values are interpreted as a \textit{space}
\end{itemize}
 


\end{flowtext}

\chapter{Firmware structure}

\section{Memory usage}\label{memory}

The firmware stores the following data:
\begin{itemize}
  \item Application code: starting at page pointer 0
  \item Settings: 2 pages starting at page pointers 476 and 477 
  \item 100 patches: 100 pages starting at page pointer 256
  \item Current panel patch data: one page at page pointer 475
  \item Sequencer data (track 1 and 2): 1 page each at pointers 456 (track 1) and 457 (track 2)
\end{itemize}

Thedata is written into application part of the storage, into the RWW (read-while-write) section of the memory. The number of pages used a firmward version can be obtained by counting the pages contained in the SysEx file.

Note to developers: the application code must fit into the first 256 pages (up to and including page pointer 255). This corresponds to a SysEx size of 85258 kB which is the equivalent of 65536 kB hex file. Code beyond this size will overwrite patches and will be corrupted by patche storage operations.


\section{Firmware SysEx structure}\label{sysex}

The firmware SysEx has the following structure. It created from the hex file (the result of the AVR complilation) by the Python script\footnote{As of the writing of this manual the script complies with Python 2 but not the newer Python 3. So depending on your local Python installation the execution inside the Makefile may need to be adjusted to ensure that script is run in Python 2.} \textit{fw2syx.py} (fw2syx subfolder). The conversion of 4 full bytes to 5 MIDI bytes is done in the same way as for patches (as described in this document). As for patch data there is one complete SysEx command per flash storage page. In case of the firmware, an additional check sum is included for each page. A special termination SysEx command is used to signal the firmware that the data stream is complete. The firmware SysEx is received and decoded in the function \textit{updater\_main()} in the class \textit{firmware.c} (firmware subfolder). 

\begin{itemize}
  \item Chain of data SysEx block, each representing a storage page in the Teensy++ 2.0
  \subitem "F0": SysEx start
  \subitem Prophet 600 signature: "00 61 16"
  \subitem Prophet 600 firmware message type: "6B"
  \subitem Pagesize (constant: "02 00" = 512 symbols)
  \subitem Page ID: one byte starting from highest ID
  \subitem 64 sets of 5 MIDI bytes (14 bit) (5 MIDI bytes are converted to 4 full bytes, this a total of 256 bytes for that page).
  \subitem 3 bytes of checksum
  \subitem "F7": SysEx end
  \item One termination SysEx
  \subitem "F0": SysEx start
  \subitem Prophet 600 signature: "00 61 16"
  \subitem Prophet 600 firmware message type: "6B"
  \subitem "00 00", instead of page size, signifying the end of the data stream
  \subitem "F7": SysEx end  
\end{itemize}


\textbf{Checksum}

(this needs to be described)


\section{Timer interrupt}\label{timeinterrupt}

\section{UART interrupt}\label{uartinterrupt}

\section{Multiplexer / DAC}\label{potmux}

\section{Mix and Drive vs. Volume A and Volum B}

\textbf{Mapping from mix and drive to volumes} 

When $drive\leq 1/2$:
\begin{eqnarray}
VolA=2\times drive \times (1-mix) \\
VolB=2\times drive \times mix 
\end{eqnarray}

When $drive> 1/2$ and $mix \leq 1/2$:
\begin{eqnarray}
VolA=1-2\times mix \times (1-drive) \\
VolB=2\times mix \times drive 
\end{eqnarray}

When $drive> 1/2$ and $mix > 1/2$:
\begin{eqnarray}
VolA=2\times drive \times (1-mix) \\
VolB=1-2\times (1-mix) \times (1-drive) 
\end{eqnarray}

\textbf{Mapping from volumes to mix and drive} 

When $VolA+VolB\leq 1$:
\begin{eqnarray}
  mix=\frac{VolB}{VolA+VolB} \\
  drive=\frac{VolA+VolB}{2} 
\end{eqnarray}

When $VolA+VolB>1$ then $mix = (1+VolB-VolA)/2$ and
\begin{eqnarray}
  VolA \geq VolB \to drive=\frac{VolB}{1+VolB-VolA} \\
  VolA < VolB \to drive=\frac{VolA}{1+VolA-VolB}
\end{eqnarray}

It is important to manage the singularities at $VolA=0$ and $VolB=0$. The following solution is chosen. In the regime $VolA+VolB>1$ the smaller of the two volumes is cut-off at a small value so that the denominator is well behaved.


\begin{thebibliography}{50} 

\bibitem{gligli}GliGli Projects Page, http://gliglisynth.blogspot.com/
\bibitem{curtis}Curtis Electro Music on Webarchive, https://web.archive.org/web/20130502132536/http://curtiselectromusic.com/Customers\_and\_Instruments.html
\bibitem{versiontwo}GliGli upgrade release 2.0,  https://gligli.github.io/p600fw/
\bibitem{modinstructions} Prophet600 Modification Instructions
\bibitem{projectsite} Project Site: http://gligli.github.io/p600fw/
\bibitem{repository} Repository: https://github.com/gligli/p600fw/
\bibitem{teensy}Teensy Site: http://www.pjrc.com/teensy/index.html
\bibitem{p600siownersmanual}Prophet 600 Owner's Manual, Sequential Circuits
\bibitem{p600siservicemanual} Prophet 600 Service Manual, Sequential Circuits

\end{thebibliography}

\end{document}

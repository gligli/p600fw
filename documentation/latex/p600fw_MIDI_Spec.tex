\documentclass[landscape, 11pt, oneside]{report}
\usepackage[a4paper, left=2cm, right=2cm, top=2.5cm]{geometry}
\usepackage{lmodern}
\usepackage{hyperref}
\usepackage{titlesec}
\usepackage{scrextend}
\usepackage{array}
\usepackage{ProphetPanel}
\usepackage{fontspec}
\usepackage[T1]{fontenc}
\usepackage{longtable}
\usepackage{ProphetTerms}
\usepackage{float}
%\usepackage[inline]{showlabels}
\setmainfont[
 BoldFont={FiraSans-Bold}, 
 ItalicFont={FiraSans-Italic},
 ]{FiraSans}
\setlength{\parindent}{0pt}
\setlength{\parskip}{1em}
\setlength{\leftmargin}{-4cm}
%\setlength{\paperheight}{210mm}
%\setlength{\paperwidth}{297mm}
\setlength{\textwidth}{25cm}
\setlength{\footskip}{2.5cm}
%% globals
\newcommand{\version}{2.X}
\newcommand{\paramgeneral}[1]{\textbf{#1}}
\newcommand{\button}[1]{\textbf{#1}}
\newcommand{\patchparameter}[2]{\paramgeneral{#2 - #1}}

% additional patch parameters
\newcommand{\clockspeed}{\patchparameter{0}{Clock/Speed}}
\newcommand{\lfoshape}{\patchparameter{1}{LFO Shape}}
\newcommand{\lfotarget}{\patchparameter{11}{LFO Target}}
\newcommand{\lfosync}{\patchparameter{111}{LFO Sync}}
\newcommand{\vibspeed}{\patchparameter{2}{Vibrato Speed}}
\newcommand{\lforange}{\patchparameter{22}{LFO Frequency Range}}
\newcommand{\vibamt}{\patchparameter{3}{Vibrato Amount}}
\newcommand{\modwheelaction}{\patchparameter{33}{Modulation Wheel Action}}
\newcommand{\moddelay}{\patchparameter{4}{Modulation Delay}}
\newcommand{\modwheeltarget}{\patchparameter{44}{Modulation Wheel Target}}
\newcommand{\2ndenv}{\patchparameter{5}{2nd Envelope Shape}}
\newcommand{\filenv}{\patchparameter{55}{Filter Envelope Shape}}
\newcommand{\envrouting}{\patchparameter{555}{Envelope Routing}}
\newcommand{\bentarget}{\patchparameter{6}{Bender Target}}
\newcommand{\bendrange}{\patchparameter{66}{Bender Target}}
\newcommand{\glide}{\patchparameter{7}{Glide Time}}
\newcommand{\assignprio}{\patchparameter{77}{Assigner Priority}}
\newcommand{\detune}{\patchparameter{8}{Unison Detune Amount}}
\newcommand{\oscpitchmode}{\patchparameter{88}{Oscillator Pitch Mode}}
\newcommand{\spread}{\patchparameter{888}{Spread Amount}}
\newcommand{\ampvel}{\patchparameter{9}{Amplitude Velocity}}
\newcommand{\filvel}{\patchparameter{99}{Filter Velocity}}
\newcommand{\syncbug}{\patchparameter{999}{Sync Bug Switch}}


% buttons
\newcommand{\numberpad}[1]{\textbf{button #1}}
\newcommand{\totape}{\button{To Tape}}
\newcommand{\fromtape}{\button{From Tape}}
\newcommand{\tune}{\button{Tune}}
\newcommand{\preset}{\button{Preset}}
\newcommand{\record}{\button{Record}}
\newcommand{\seq1}{\button{Seq 1}}
\newcommand{\seq2}{\button{Seq 2}}
\newcommand{\arpud}{\button{Arp U/D}}
\newcommand{\arpass}{\button{Arp Assign}}

% panel controls
\newcommand{\polymodenv}{\patchparameter{Poly-Mod}{Envelope}}
\newcommand{\polymodosc}{\patchparameter{Poly-Mod}{Oscillator B}}
\newcommand{\polymodfreq}{\patchparameter{Poly-Mod}{Frequency Target}}
\newcommand{\polymodfilter}{\patchparameter{Poly-Mod}{Filter Target}}

%\newenvironment{flowtext}{\addmargin[0cm]{7cm}}{\endaddmargin} % standard text width (reduced for layout)

\title{MIDI Specification for the 2022 upgraded Prophet-600}
\pagestyle{myheadings}\markright{Technical Reference for the GliGli upgraded Prophet-600}
\makeatletter
\renewcommand\chapter{\pagestyle{myheadings}\markright{Technical Reference for the GliGli upgraded Prophet-600}\global\@topnum\z@\@afterindentfalse\secdef\@chapter\@schapter}
\makeatother
\titleformat{\chapter}[display]{\LARGE\bfseries}{}{0.0cm}{}
\titlespacing{\chapter}{0pt}{*0}{*0}
%\titleformat{\section}[display]{\Large\bfseries}{}{0.0cm}{}
\titlespacing{\section}{0pt}{*0}{*0}

\begin{document}


% this it the title "Prophet-600"
  \begin{tikzpicture}[scale=0.7]
    \begin{scope}[xslant=0.1]
        \SSGBit[1.5cm]{0,0}{12567}
        \SSGBit[1.5cm]{2.5cm,0}{57}
        \SSGBit[1.5cm]{5cm,0}{5734}
        \SSGBit[1.5cm]{7.5,0}{12567}
        \SSGBit[1.5cm]{10cm,0}{3567}
        \SSGBit[1.5cm]{12.5cm,0}{14567}
        \SSGBit[1.5cm]{15cm,0}{4567}
        \SSGBit[1.5cm]{17.5cm,0}{7}
        \SSGBit[1.5cm]{20cm,0}{134567}
        \SSGBit[1.5cm]{22.5cm,0}{123456}
        \SSGBit[1.5cm]{25cm,0}{123456}
      \end{scope}
  \end{tikzpicture}

\vspace{1cm}
  
  \Large
  MIDI Specification for the Prophet-600 GliGli based firmware \vspace{0.4cm} \\
  \large
  Edition: 1.0 (8.8.2022) \vspace{0.3cm}\\
  Refers to firmware version: \version \vspace{0.3cm}\\
  Refers to storage version: 8
  \normalsize

%\maketitle
%\tableofcontents

\chapter{Preface}

The purpose of this technical reference is to provide reliable reference material for people implementing interfaces to/from the Prophet-600, for example a MIDI SysEx patch editor or a MIDI CC patch editor. Although designed to be helpful and drafted with care, this document comes with no warranty for completeness and correctness. If you find that something is wrong or missing or if you have an idea what should be documented, please get in contact via github (https://github.com/image-et-son/p600fw).

\pagebreak

\chapter{MIDI SysEx Patch Data Implementation}

\section{MIDI bytes to full byte conversion}\label{midibyteconversion}

The data stream embedded in Prophet-600 MIDI SysEx patch commands consist of blocks of 5 MIDI bytes which encode 4 full bytes\footnote{This encoding is commonly used but by no means universal. Still, I could find no reference or standard source for this encoding and also no commonly used designation.}. Each MIDI byte can hold values up to 127 since the 8th bit is reserved. In a MIDI data block of 5 bytes the first 4 MIDI bytes contain the first (lowest) 7 bits of the 4 data bytes to be encoded. The 8th bits of the 4 data bytes are encoded in the first (lowest) 4 bits of the 5th MIDI byte.   

Examples:

\begin{itemize}
  \item "01 00 00 00" is encoded in SysEx MIDI as "01 00 00 00 00"
  \item "80 00 00 00" is encoded in SysEx MIDI as "00 00 00 00 08"
  \item "FF 00 00 00" is encoded in SysEx MIDI as "7F 00 00 00 08"
  \item "00 00 00 FF" is encoded in SysEx MIDI as "00 00 00 7F 01"
  \item "FF FF FF FF" is encoded in SysEx MIDI as "7F 7F 7F 7F 0F"
\end{itemize}
 
Note, however, that double byte parameters (which is applies to most continuous patch parameters in the firmware) are sequenced such that in the full bytes sequence the least significant byte (LSB) comes before the most significant byte (MSB). 

Note: patch parameters  consist only of single byte parameters (typically stepped parameters) or double byte parameters (all continuous parameters). However, in the patch structure overhead there are longer byte sequences: the unique Prophet-600 SysEx command identifier (3 byte hex value: 00 61 16) and the Prophet-600 storage page initialization number (4 byte hex value: 00 61 16 A5, referred to as \textit{Storage Magic} in the code). 

\section{Patch SysEx structure}\label{sysexpatch}

Patch parameters are stored as patches in each of the 100 slots on the Teensy++ 2.0 board (see section on memory usage). Apart from technical overhead this covers:
\begin{itemize}
  \item panel control settings
  \item menu parameters
  \item special data: per note tuning
  \item special data: note pattern (when the patch has unison activated with a chord pattern, this data contains the latched pattern) 
  \item special data: patch name 
\end{itemize}

The SysEx MIDI data is build directly from the storage data. It therefore has the same structure (order of parameters). 

\subsection{Storage version}

Due to extensions of the parameter scope for additional functions, the patch storage structure depends on the firmware version. The storage versioning is as follows:

\begin{itemize}
  \item Firmware version 2.0: storage version 2
  \item Firmware version 2.1 RC3: storage version 7
  \item Firmware version \version: storage version 8  
\end{itemize}

For the implementation of an MIDI SysEx based patch editor it is crucial to take storage version into account. It is not recommended to try to convert a lower version SysEx file to a higher version, as several parameters have the value ranges changed. Instead one should always import a lower version SysEx patch file to the Prophet-600 with a newer firmware installed that uses the storage version you want and then export it again. The exported SysEx patch file will be properly converted. 

Note: it would be possible to document the transformations / remappings between versions 7 and 8. If you feel you need this for what you are trying to do, please get in contact via https://github.com/image-et-son/p600fw.

\subsection{Patch MIDI SysEx structure}

The SysEx structure for patch data is as follows:

\begin{enumerate}
  \item "F0": SysEx start
  \item Prophet-600 SysEx command signature: "00 61 16"
  \item Prophet-600 SysEx patch dump type: "01"
  \item Patch data section: series of MIDI data blocks, each consisting of 5 MIDI bytes (14 bit) 
  \item "F7": SysEx end
\end{enumerate}

The Prophet-600 ignores SysEx commands which do not carry the signature "00 61 16" with the exception of original Z80 produced SysEx files.

The data part of the patch SysEx has the same underlying structure as the internal storage in the sense that the parameters are stored in exactly the same order. As described in section \ref{midibyteconversion} the storage of double  byte values is added to the MIDI byte sequence in the order \textit{LSB}, \textit{MSB}. E.g the lower byte comes first. This also applies if a double byte is split across two MIDI data blocks (e.g. the \textit{LSB} is the fourth byte of one MIDI data block and the \textit{MSB} is the first byte of the next MIDI data block). 

The technical overhead inside the patch data section (item 4 in the list above) consists of another specific signature (\textit{Storage Magic}) and the stroage version. The signature is used to identify whether the data section of the SysEx represents a valid storage page of the Prophet-600. It should  always be present for any valid patch data independently from the storage version. If that signature is not present or differs the patch data will be ignored. The storage version should be used to identify the correct mapping of values.  

Note that the MIDI SysEx patch data contains all parameters up to and including the last non-zero value. It therefore has variable a size. However, it always consists of complete MIDI blocks (5 MIDI bytes which convert into 4 data bytes).

Each patch SysEx command as shown above represents exactly one flash storage page on the Teensy++ 2.0 board. A patch bank dump simply consists of a sequence of closed SysEx patch commands.

\subsection{Patch storage structure}\label{patchstore}

The following table specifies the patch data storage up to and including storage version 8. 

\footnotesize
\renewcommand{\arraystretch}{1.3}

\begin{longtable}[l]{p{5cm}|p{2cm}|p{1.5cm}|p{1.5cm}|p{2cm}|p{2.2cm}|p{5cm}} 
\textbf{Patch Data} & \textbf{Type} & \textbf{Byte} & \textbf{Count} & \textbf{Total Bytes} & \textbf{From Version} & \textbf{Stepped value} \\ \hline
\endfirsthead
\textbf{Patch Data} & \textbf{Type} & \textbf{Byte} & \textbf{Count} & \textbf{Total Bytes} & \textbf{From Version} & \textbf{Stepped value} \\ \hline
\endhead 
Patch Number & Technical & 1 & 1 & 1 & 0 &  \\ \hline
Storage Key (aka MAGIC) & Technical & 4 & 1 & 4 & 0 &  \\ \hline
Storage Version & Technical & 1 & 1 & 1 & 0 &  \\ \hline
Fequency A & Continuous & 2 & 1 & 2 & 1 &  \\ \hline
Volume A & Continuous & 2 & 1 & 2 & 1 &  \\ \hline
PWA & Continuous & 2 & 1 & 2 & 1 &  \\ \hline
Fequency B & Continuous & 2 & 1 & 2 & 1 &  \\ \hline
Volume B & Continuous & 2 & 1 & 2 & 1 &  \\ \hline
PWB & Continuous & 2 & 1 & 2 & 1 &  \\ \hline
Frequency Fine B & Continuous & 2 & 1 & 2 & 1 &  \\ \hline
Cutoof & Continuous & 2 & 1 & 2 & 1 &  \\ \hline
Resonance & Continuous & 2 & 1 & 2 & 1 &  \\ \hline
Filter Envelope Amount & Continuous & 2 & 1 & 2 & 1 &  \\ \hline
Filter Release & Continuous & 2 & 1 & 2 & 1 &  \\ \hline
Filter Sustain & Continuous & 2 & 1 & 2 & 1 &  \\ \hline
Filter Decay & Continuous & 2 & 1 & 2 & 1 &  \\ \hline
Filter Attack & Continuous & 2 & 1 & 2 & 1 &  \\ \hline
Amp Release & Continuous & 2 & 1 & 2 & 1 &  \\ \hline
Amp Sustain & Continuous & 2 & 1 & 2 & 1 &  \\ \hline
Amp Decay & Continuous & 2 & 1 & 2 & 1 &  \\ \hline
Amp Attack & Continuous & 2 & 1 & 2 & 1 &  \\ \hline
Poly Mod Envelope Amount & Continuous & 2 & 1 & 2 & 1 &  \\ \hline
Poly Mod OSC B & Continuous & 2 & 1 & 2 & 1 &  \\ \hline
LFO Frequency & Continuous & 2 & 1 & 2 & 1 &  \\ \hline
LFO Amount & Continuous & 2 & 1 & 2 & 1 &  \\ \hline
Glide & Continuous & 2 & 1 & 2 & 1 &  \\ \hline
Amp Velocity & Continuous & 2 & 1 & 2 & 1 &  \\ \hline
Filter Velocity & Continuous & 2 & 1 & 2 & 1 &  \\ \hline
Saw A & Stepped & 1 & 1 & 1 & 1 & Off, On \\ \hline
Tri A & Stepped & 1 & 1 & 1 & 1 & Off, On \\ \hline
SQR A & Stepped & 1 & 1 & 1 & 1 & Off, On \\ \hline
Saw B & Stepped & 1 & 1 & 1 & 1 & Off, On \\ \hline
Tri B & Stepped & 1 & 1 & 1 & 1 & Off, On \\ \hline
SQR B & Stepped & 1 & 1 & 1 & 1 & Off, On \\ \hline
Sync & Stepped & 1 & 1 & 1 & 1 & Off, On \\ \hline
Poly Mod Frequency A & Stepped & 1 & 1 & 1 & 1 & Off, On \\ \hline
Poly Mod Filter & Stepped & 1 & 1 & 1 & 1 & Off, On \\ \hline
LFO Shape & Stepped & 1 & 1 & 1 & 1 & Pulse, Triangle, Random, Sin, Noise, Saw\footnote{In the patch SysEx the combined effect of the LFO shape panel switch and the additional patch parameter for further LFO shape options is combined into this single stepped parameter with 6 options. An external editor could therefore use a single 6-way parameter for the LFO shape.} \\ \hline
(unused) & n/a & 1 & 1 & 1 & n/a\footnote{This slot was used for \textit{LFO Frequency Range} from versions 1 to 7} &  \\ \hline
LFO Targets & Stepped & 1 & 1 & 1 & 1 & see section \ref{lfotarget}  \\ \hline
Tracking Shift & Stepped & 1 & 1 & 1 & 1 & Off, Half, Full \\ \hline
Filter Envelope Shape & Stepped & 1 & 1 & 1 & 1 & Linear, Exponential \\ \hline
Filter Envelope Speed & Stepped & 1 & 1 & 1 & 1 & Slow, Fast \\ \hline
Amp Envelope Shape & Stepped & 1 & 1 & 1 & 1 & Linear, Exponential \\ \hline
Amp Envelope Speed & Stepped & 1 & 1 & 1 & 1 & Slow, Fast \\ \hline
Unison & Stepped & 1 & 1 & 1 & 1 & Off, On \\ \hline
Assigner Priority & Stepped & 1 & 1 & 1 & 1 & Last, Low, High \\ \hline
Bender Semitones & Stepped & 1 & 1 & 1 & 1 & 2nd, 3rd, 5th, Oct\footnote{Note that the numerical values for this option range are: 2, 3, 5 and 8, i.e. value corresponds to the actual semitones rather than the index of the selected option.} \\ \hline
Bender Target & Stepped & 1 & 1 & 1 & 1 & Off, A \& B, VCF, Volume, B \\ \hline
Mod Wheel Shift & Stepped & 1 & 1 & 1 & 1 & Touch, Soft, High, Full \\ \hline
Chromatic Pitch & Stepped & 1 & 1 & 1 & 1 & Free, Semi,Octave, Octave-Semi, Semi-Octave, Octave-Free, Free-Octave \\ \hline
Modulation Delay & Continuous & 2 & 1 & 2 & 2 &  \\ \hline
Vibrato Frequency & Continuous & 2 & 1 & 2 & 2 &  \\ \hline
Vibrato Amount & Continuous & 2 & 1 & 2 & 2 &  \\ \hline
Unison Detune & Continuous & 2 & 1 & 2 & 2 &  \\ \hline
(unused) & n/a & 2 & 1 & 2 & n/a\footnote{This slot was used for arpeggiator/sequencer clock speed fro versions 1 to 7. From version 8 on, clock speed is consistently implemented as a setting only.} &  \\ \hline
Modulation Wheel Target & Stepped & 1 & 1 & 1 & 2 & LFO, Vibrato \\ \hline
Vibrato Target & Stepped & 1 & 1 & 1 & 2 & VCO A \& B, VCA, VCO A, VCO B \\ \hline
Voice Pattern (6 voices) & Special & 1 & 6 & 6 & 2 &  \\ \hline
Tuning per Note (12 notes) & Special & 2 & 12 & 24 & 7 &  \\ \hline
Sync Bug & Stepped & 1 & 1 & 1 & 8 & Off, On \\ \hline
Vintage (Spread) & Continuous & 2 & 1 & 2 & 8 &  \\ \hline
External CV & Continuous & 2 & 1 & 2 & 8 &  \\ \hline
Envelope Routing & Stepped & 1 & 1 & 1 & 8 & Standard, Poly-Amp, Poly, Gate \\ \hline
Voice Assign & Stepped & 1 & 1 & 1 & 8 & First, Cycle \\ \hline
LFO Sync & Stepped & 1 & 1 & 1 & 8 & Off, Key, 1, 2, 3, 4, 5, 6, 8 \\ \hline
Patch Name & Technical & 1 & 16 & 16 & 8 &  \\ \hline




\end{longtable}


\textbf{Example}

Consider building the SysEx patch files for patch number 5 from storage in the latest version 8. The first 8 data bytes prior to conversion are: "05 00 61 16 A5 08 00 FF", where 05 is the patch number, "00 61 16 A5" is the \textit{Storage Magic}, "08" is the storage version and "00 FF" is an example value for the first real patch parameter which is frequency of oscillator A. These 8 data bytes are first rearranged into two blocks of 4 bytes like: "05 A5 16 61" "00 08 FF 00". Note that the order of bytes for each data block is reversed with \textit{LSB} first, etc. Then the conversion to 5 MIDI bytes leads to two blocks of 5 MIDI bytes like this: "05 25 16 61 02" "00 08 7F 00 04".


\subsection{Patch Name}

Up to version \version the Prophet-600 does no display the patch name and offers no option the change the patch name. However, the patch name data is stored in the patch and is preserved when the patch is send to Prophet-600 via MIDI SysEx, then stored on the unit and then exported again in SysEx format. An external SysEx patch editor could therefore make use of the patch name.

The patch consists of 16 bytes and the enconding is UTF-8 with printable characters as follows:

\begin{itemize}
  \item 00: NULL character terminates the patch name
  \item 01 - 0F: non-printable characters are interpreted as \textit{space}
  \item 20 - 7E: standard UTF-8 printable charatcers from \textit{space} through number, letters and special characters up to \textasciitilde
  \item 7F - FF: DEL and following values are interpreted as a \textit{space}
\end{itemize}
 


\section{Tools}

The release comes with a Python script which converts Prophet-600 Patch MIDI SysEx to plain values. This is a specialist tool (which came in handy during the testing phase) but can also be the starting point for the development of a proper patch management tool. The following rfers to the (still rudimentary) version 1 of this tool.

The tool consists of the script \textit{syx\_converter\_v1.py} and the two configuration files \textit{storage\_7.spec} and \textit{storage\_8.spec}. The script expects the two configuration files in the folder. To use the tool you need to have Python 3 installed. The command line usage would be \textit{>py syx\_converter\_v1.py <filename> [-p<patchnumber>]}. The filename (or path) must be provided. It is the SysEx File to be converted. The parameter \textit{p} is optional. When provided and \textit{<patchnumber>} is between 0 and 99, then the script will try to extract that patch from the file. If the SysEx file does not contain that patch number the output is empty.

The script lists the parameters in the SysEx is a simple way as shown here for patch in slot 20 (only parts of the the result):

\begin{itemize}
  \item[>] Patch Number:  20 
  \item[>] Storage version is 8
  \item[>] Frequency A :  59776
  \item[>] Volume A :  52224
  \item[>] PWA :  24000 
  \item[>] Frequency B :  7408 
  \item[>] Volume B :  54528
  \item[>] PWB :  52416
  \item[>] Frequency Fine B :  40344
  \item[>] Cutoff :  240
  \item[>] Resonance :  7936
  \item[>] Filter Envelope Amount :  48538
  \item[>] Filter Release :  54016
  \item[>] ...  
  \item[>] Bender Semitones :  3
  \item[>] Bender Target :  1
  \item[>] ...  
\end{itemize}

The continuous parameter have a value range of 0 to 65535 (16 bit resolution). Up to version 1 of the script, the values of stepped parameters are not name resolved but instead the index of the choice is displayed. In the example above, the bender semitones "3" means 3 semitones but the bender target value "1" means \textit{AB} (pitch bend of oscillators A and B) which is the second value in the choice list. The indexing always starts at "0". There are also some special parameters as described in this document.
 
The SysEx specification for version 7 (corresponds to version 2.1 RC3) is supported.

\chapter{MIDI CC Implementation}

Two types of MIDI events are implemented, e.g. the upgraded Prophet 600 can receive and apply them:

\begin{itemize}
  \setlength\itemsep{0cm}
  \item MIDI CC Continuous parameters: value resolution of 0-16383 using 2 CCs (fine and coarse), or value resolution of 0-127 using only the coarse one
  \item MIDI CC Stepped parameters: 0-127, variable number of steps. They work by dividing the 0-127 range in as many zones as there are choices for the parameter. E.g.: "Unison" is an on-off parameter and therefore has s choices. In this case it is \textit{off} for 0-63 and it is \textbf{on} for 64-127.
  \item Performance events (note events, pitch bend)
  \item Real time events
  \item SysEx (for different purposes)
\end{itemize}

The Prophet 600 receives Continuous Controllers (CC) in \presetmode only. 

\textbf{MIDI CC Patch Parameters} 

\footnotesize
\renewcommand{\arraystretch}{1.3}

\begin{longtable}[l]{ p{5cm}|p{2cm}|p{1.5cm}|p{1.5cm}|p{5cm}|p{2cm}|p{1cm}} 
\textbf{Continuous Parameter} & \textbf{Type} & \textbf{CC Coarse} & \textbf{CC Fine} & \textbf{Stepped Parameter} & \textbf{Type} & \textbf{CC} \\ \hline
\endfirsthead
\textbf{Continuous Parameter} & \textbf{Type} & \textbf{CC Coarse} & \textbf{CC Fine} & \textbf{Stepped Parameter} & \textbf{Type} & \textbf{CC} \\ \hline
\endhead 
Osc A Frequency & Continuous & 16 & 80 & Osc A Saw & Stepped & 48 \\ \hline
Osc A Volume & Continuous & 17 & 81 & Osc A Triangle & Stepped & 49 \\ \hline
Osc A Pulse Width & Continuous & 18 & 82 & Osc A Square & Stepped & 50 \\ \hline
Osc B Frequency & Continuous & 19 & 83 & Osc B Saw & Stepped & 51 \\ \hline
Osc B Volume & Continuous & 20 & 84 & Osc B Triangle & Stepped & 52 \\ \hline
Osc B Pulse Width & Continuous & 21 & 85 & Osc B Sqr & Stepped & 53 \\ \hline
Osc B Fine & Continuous & 22 & 86 & Sync & Stepped & 54 \\ \hline
Cutoff & Continuous & 23 & 87 & Poly Mod Oscillator A Destination & Stepped & 55 \\ \hline
Resonance & Continuous & 24 & 88 & Poly Mod Filter Destination & Stepped & 56 \\ \hline
Filter Envelope Amount & Continuous & 25 & 89 & LFO Shape & Stepped & 57 \\ \hline
Filter Release  &  Continuous  & 26 & 90 &  LFO Targets  &  Stepped  &  59 \\ \hline
Filter Sustain  &  Continuous  & 27 & 91 &  Keyboard Filter Tracking  &  Stepped  &  60 \\ \hline
Filter Decay  &  Continuous  & 28 & 92 &  Filter Envelope Shape  &  Stepped  &  61 \\ \hline
Filter Attack  &  Continuous  & 29 & 93 &  Filter Envelope Fast/Slow  &  Stepped  &  62 \\ \hline
2nd Release  &  Continuous  & 30 & 94 &  2nd Envelope Shape  &  Stepped  &  63 \\ \hline
2nd Sustain  &  Continuous  & 31 & 95 &  Unison  &  Stepped  &  65 \\ \hline
2nd Decay  &  Continuous  & 32 & 96 &  Assigner Priority Mode  &  Stepped  &  66 \\ \hline
2nd Attack  &  Continuous  & 33 & 97 &  Bender Range (semitones)  &  Stepped  &  67 \\ \hline
Poly Mod Filter Amount  &  Continuous  & 34 & 98 &  Bender Target  &  Stepped  &  68 \\ \hline
Poly Mod Osc B Amount  &  Continuous  & 35 & 99 &  Mod Wheel Range  &  Stepped  &  69 \\ \hline
LFO Frequency  &  Continuous  & 36 & 100 &  Osc pitch mode  &  Stepped  &  70 \\ \hline
LFO Amount  &  Continuous  & 37 & 101 &  Mod Wheel Target  &  Stepped  &  71 \\ \hline
Glide  &  Continuous  & 38 & 102 &  Vibrato Target  &  Stepped  &  72 \\ \hline
Amp Velocity  &  Continuous  & 39 & 103 &   2nd Envelope Fast/Slow  &  Stepped  &  73 \\ \hline
Filter Velocity  &  Continuous  & 40 & 104 &  Sync Bug  &  Stepped  &  74 \\ \hline
Modulation delay  &  Continuous  & 41 & 105 &  Voice Assign  &  Stepped  &  75 \\ \hline
Vibrato frequency  &  Continuous  & 42 & 106 &  Envelope Routing  &  Stepped  &  76 \\ \hline
Vibrato amount  &  Continuous  & 43 & 107 &  LFO Sync  &  Stepped  &  77 \\ \hline
Unison detune  &  Continuous  & 44 & 108 &    &    &   \\ \hline
External CV amount  &  Continuous  & 46 & 110 &  &  &   \\ \hline 
Vintage / Spread  &  Continuous  & 47 & 111 &  &  &   \\ \hline 
 
\end{longtable}

\normalsize

\textbf{Change in MIDI CC from version 2.1 RC3 to \version}

The following MIDI CC have been changed.

\begin{itemize}
  \item For the filter envelope release and decay and the 2nd envelope release and decay the time value range has been rescaled for the \textit{linear} envelope shape. This was an unavoidable side effect of changing the strictly linear shapes to linear shapes with soft tails. The exponential envelopes are unaffected.
  \item The \lfofreq has been totally rescaled. The stepped frequency range parameter has been omitted and the value range of the \lfofreq dial now covers the entire frequency range. However, the dial to frequency relation has also been changed from linear to exponential, making it much softer at low settings. Consequently, the MIDI CC support for the LFO frequency range (up to firmware version 2.1 RC3) has been omitted. In order to avoid potential conflicts with MIDI tools developed for firmware versions 2.0 and 2.1 RC3, the legacy MIDI CC 58 has not been re-designated. MIDI CC 58 is ignored.
  \item The value to frequency relation for the \vibspeed has been changed from linear to exponential.   
  \item The \modwheelrange has been reconfigured with new value which also produce different wheel actions.
\end{itemize}

\textbf{Other MIDI CC Settings}

\begin{longtable}[l]{ p{6cm}|p{2.5cm}|p{1.5cm}|p{2cm}} 
\textbf{Parameter} & \textbf{Type} & \textbf{CC} & \textbf{CC Fine} \\ \hline
Toggle Live/Preset Mode & Stepped & 0 & \\ \hline
Mod Wheel & Continuous & 1 & \\ \hline
Master Volume & Continuous & 7 & \\ \hline
Sequencer / arpeggiator clock & Continuous & 45 & 109 \\ \hline
Hold Pedal & Stepped & 64 & \\ \hline
Local On/Off & Stepped & 120 & \\ \hline
All Notes Off & Stepped & 123 & \\ \hline
\end{longtable}



\normalsize


\end{document}

For supporting the integration of the Prophet 600 in a studio or live environment using MIDI, several MIDI functions are supported.

\begin{itemize}
  \item Note send and receive
  \item Control of many parameters using MIDI CC
  \item Local on and local off modes
  \item SysEx support for patch management (for this see section \ref{patchmgmt})
\end{itemize}

The MIDI send and receive channels can be set using dedicated miscellaneous settings as follows. In \shiftmode or \shiftlock:

\begin{enumerate}
  \setlength\itemsep{0cm}
  \item \numberbut{1} cycles through the MIDI receive channels, from \textit{OMNI} through channels \textit{Ch1} to \textit{Ch16}
  \item \numberbut{2}  cycles through the MIDI send channels, from \textit{Ch1} to \textit{Ch16}
\end{enumerate}

\textbf{Notes, performance and MIDI CC}

The Prophet 600 sends the following MIDI signals. 

\begin{enumerate}
  \setlength\itemsep{0cm}
  \item Key on, key off in all modes
  \item Pitch bender
  \item Hold pedal (in polyphonic mode, not in unison/chord mode or running arpeggiator or sequencer)
  \item Modulation wheel (technically a MIDI CC event)
  \item Program change (technically a MIDI CC event)
\end{enumerate}

All of the events can also be received and executed with the following exceptions/modifications.

\begin{enumerate}
  \setlength\itemsep{0cm}
  \item Unlike the internal key on/off events, in arpeggiator mode incoming MIDI key on/off are not entered into arpeggiator but play normally 
  \item Unlike the internal key on/off events, in \shiftmode or \shiftlock incoming MIDI notes do no cause transpositions (but are ignored) 
  \item External MIDI pitch bend is added to the pitch bend from the on-board bender 
  \item MIDI Hold Pedal events are also applied in unison/chord mode or running arpeggiator or sequencer where it has the effect of latching notes/chords
  \item MIDI program change messages are only applied in \presetmode and are ignored otherwise  
\end{enumerate}

The instrument does not send MIDI CC (apart from modulation wheel and program change). However, the Prophet 600 supports a comprehensive list of general and specific MIDI CC events. For reference see section \ref{midiimplementation}. Incoming MIDI CC events are only applied in \presetmode.

\textbf{Local off mode}

The Prophet 600 supports a \textit{local off} mode which is toggled using the setting at \numberbut{0}. In \textit{local off} various internal events (such as notes) are not executed in the Prophet 600 directly but they are still sent as MIDI events. The typical setup would be to connect to a MIDI recorder and feed the recorded MIDI messages back to the instrument for replay. The \textit{local off} mode ensures that there is no doubling of events.

Note that in \textit{local on} mode external MIDI does not play into the arpeggiator but plays over the arpeggiator while the local keyboard plays into the arpeggiator.

\textbf{Overview of MIDI integration}

\begin{table}[H]
  \begin{tabular}{lcccccc}
     &
      \multicolumn{3}{c}{\textit{local on}} &
      \multicolumn{3}{c}{\textit{local off}} \\   \cline{2-7} 
    & Apply & Send MIDI & Apply MIDI & Apply & Send MIDI & Apply MIDI \\
    \hline
      \vline Keys on/off playing	& Yes & Yes	& Yes & No & Yes & Yes \\
      Keys into arpeggiator	& Yes & Yes	& Played & No & Yes & Yes \\
      Keys into sequencer	& Yes & Yes	& Yes & No & Yes & Yes \\
      Modulation Wheel	& Yes & Yes	& Yes & No & Yes & Yes \\
      Pitch Bend	& Yes & Yes	& Added & No & Yes & Yes \\
      Program Change	& Yes & Yes	& in preset mode & Yes & Yes & in preset mode \\
      Pedal Hold (Latch)	& Yes & No & No & Yes & No & No \\
      Pedal Hold (Sustain)	& Yes & Yes & Yes & No & Yes & Yes \\
      Other MIDI CC	& Yes & No & in preset mode & Yes & No & in preset mode \\
  \end{tabular}
\end{table}
